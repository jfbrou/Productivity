\documentclass[11pt]{article}

% ── Typography ────────────────────────────────────────────────────────
\usepackage[T1]{fontenc}
\usepackage{mathpazo}                          % Palatino serif body
\usepackage[scaled=0.92]{helvet}               % Helvetica for headings
\usepackage{microtype}
\linespread{1.25}
\setlength{\parindent}{0pt}
\setlength{\parskip}{6pt}

% ── Page geometry ─────────────────────────────────────────────────────
\usepackage[margin=2.8cm, headheight=18pt]{geometry}

% ── Colours ───────────────────────────────────────────────────────────
\usepackage{xcolor}
\definecolor{HECnavy}{HTML}{002855}
\definecolor{HECcyan}{HTML}{00AEC7}
\definecolor{HECgray}{HTML}{D9D9D6}
\definecolor{HECcoolgray}{HTML}{888B8D}

% ── Graphics & TikZ ──────────────────────────────────────────────────
\usepackage{graphicx}
\graphicspath{{../Figures/}}
\usepackage{tikz}

% ── Tables ────────────────────────────────────────────────────────────
\usepackage{booktabs, array}
\newcommand*\includetables[1]{\input{../Tables/#1.tex}}

% ── Captions ──────────────────────────────────────────────────────────
\usepackage{caption}
\captionsetup{
  font=small,
  labelfont={bf,color=HECnavy,sf},
  textfont={sf},
  labelsep=endash,
  justification=raggedright,
  singlelinecheck=false
}

% ── Floats ────────────────────────────────────────────────────────────
\usepackage{float}

% ── Boxes ─────────────────────────────────────────────────────────────
\usepackage[most]{tcolorbox}

% Executive summary box: faint navy tint, left border
\newtcolorbox{execbox}{
  colback=HECnavy!5,
  colframe=HECnavy,
  boxrule=0pt,
  leftrule=2pt,
  arc=0pt,
  outer arc=0pt,
  left=10pt,
  right=10pt,
  top=8pt,
  bottom=8pt,
  fonttitle=\sffamily\scshape\color{HECnavy},
  title=Faits saillants,
  toptitle=4pt,
  bottomtitle=4pt
}

% Key finding box: no background, thin cyan left border
\newtcolorbox{keyfinding}{
  colback=white,
  colframe=HECcyan,
  boxrule=0pt,
  leftrule=1.5pt,
  arc=0pt,
  outer arc=0pt,
  left=10pt,
  right=0pt,
  top=4pt,
  bottom=4pt
}

% ── Section formatting ────────────────────────────────────────────────
\usepackage{titlesec}

\titleformat{\section}
  {\Large\bfseries\sffamily\color{HECnavy}}
  {}
  {0pt}
  {}

\titlespacing*{\section}{0pt}{28pt}{10pt}

\titleformat{\subsection}
  {\large\bfseries\sffamily\color{HECnavy}}
  {\thesubsection}
  {0.5em}
  {}

\titlespacing*{\subsection}{0pt}{18pt}{6pt}

% ── Headers & footers ────────────────────────────────────────────────
\usepackage{fancyhdr}
\pagestyle{fancy}
\fancyhf{}
\fancyhead[L]{%
  {\small\sffamily\scshape\color{HECnavy}CPP}%
  \enspace{\color{HECgray}\textbar}\enspace%
  {\small\sffamily\color{HECcoolgray}HEC Montr\'eal}%
}
\fancyhead[R]{\small\sffamily\color{HECcoolgray}\thepage}
\fancyfoot{}

% Thin rule below the header
\renewcommand{\headrule}{%
  \vspace{2pt}%
  {\color{HECgray}\hrule height 0.4pt}%
}

% ── Hyperlinks ────────────────────────────────────────────────────────
\usepackage[colorlinks, allcolors=HECnavy]{hyperref}
\urlstyle{rm}

% ── Bibliography ──────────────────────────────────────────────────────
\usepackage[longnamesfirst]{natbib}

% ── Language ──────────────────────────────────────────────────────────
\usepackage[french]{babel}

% ── Mathematics ───────────────────────────────────────────────────────
\usepackage{amsmath}

% ── Appendix ──────────────────────────────────────────────────────────
\usepackage{appendix}

% ══════════════════════════════════════════════════════════════════════
%  DOCUMENT
% ══════════════════════════════════════════════════════════════════════
\begin{document}

% ── Title page ────────────────────────────────────────────────────────
\thispagestyle{empty}

\begin{tikzpicture}[remember picture, overlay]
  \fill[HECnavy]
    (current page.north west) rectangle
    ([yshift=-2.2cm]current page.north east);
  \node[anchor=west, text=white, font=\sffamily\scshape\small]
    at ([xshift=2.8cm, yshift=-1.1cm]current page.north west)
    {\textls[80]{Centre sur la productivit\'e et la prosp\'erit\'e}};
  \node[anchor=east, text=white, font=\sffamily\small]
    at ([xshift=-2.8cm, yshift=-1.1cm]current page.north east)
    {HEC Montr\'eal};
\end{tikzpicture}

\vspace*{3.5cm}

{\raggedright
\fontsize{26}{31}\selectfont\sffamily\bfseries\color{HECnavy}
D\'ecomposition de la croissance\\[4pt]
de la productivit\'e du travail\\[4pt]
au Canada, 1961--2019\par}

\vspace{14pt}

{\color{HECcyan}\rule{4cm}{1.5pt}}

\vspace{14pt}

{\sffamily\scshape\color{HECnavy} Note de recherche}

\vspace{20pt}

{\color{HECcoolgray}\sffamily
\href{https://www.jeanfelixbrouillette.com}{Jean-F\'elix Brouillette}\\[2pt]
\href{mailto:jean-felix.brouillette@hec.ca}{jean-felix.brouillette@hec.ca}\\[2pt]
F\'evrier 2026}

\clearpage

% ── Executive summary ─────────────────────────────────────────────────

\begin{execbox}
\begin{itemize}\sffamily\small
  \setlength{\itemsep}{3pt}
  \item La croissance de la productivit\'e du travail au Canada a chut\'e de 2\,\% par ann\'ee (1961--1980) \`a 0.5\,\% (2000--2019).
  \item Ce ralentissement s'explique presque enti\`erement par un effondrement de la productivit\'e totale des facteurs (PTF), et non par un manque d'investissement.
  \item L'effet Baumol --- le d\'eplacement de l'activit\'e vers des secteurs \`a faible productivit\'e --- est pr\'esent mais modeste. C'est la PTF \textit{au sein} des industries qui stagne.
  \item La m\'ethode repose sur des identit\'es comptables et des donn\'ees publiques de Statistique Canada. Elle s'applique \`a tout pays disposant de comptes de productivit\'e sectoriels.
\end{itemize}
\end{execbox}

\vspace{6pt}

% ══════════════════════════════════════════════════════════════════════
%  MAIN TEXT
% ══════════════════════════════════════════════════════════════════════

\section*{Le ralentissement de la productivit\'e au Canada}

Entre 1961 et 1980, la productivit\'e du travail au Canada progressait de 2\,\% par ann\'ee. Depuis 2000, ce rythme est tomb\'e \`a 0.5\,\%. Ce ralentissement est au c\oe{}ur du d\'ebat sur le niveau de vie des Canadiens\,: la productivit\'e du travail d\'etermine, \`a long terme, la croissance des salaires r\'eels et du revenu par habitant.

Ce document d\'ecompose la productivit\'e du travail agr\'eg\'ee en deux \'etapes. La premi\`ere s\'epare la contribution de la productivit\'e totale des facteurs (PTF) de celle de l'approfondissement du capital. La seconde d\'ecompose la PTF agr\'eg\'ee en une composante intra-sectorielle et un effet de composition (effet Baumol). L'approche repose sur des identit\'es comptables et des donn\'ees sectorielles publiques de Statistique Canada, couvrant 38 industries de 1961 \`a 2019.

\begin{keyfinding}
\itshape Le ralentissement de la productivit\'e du travail depuis 2000 s'explique presque enti\`erement par la PTF, et non par un d\'eficit d'investissement.
\end{keyfinding}

% ══════════════════════════════════════════════════════════════════════

\section*{PTF et investissement}

La productivit\'e du travail ($Y/L$) peut augmenter de deux fa\c{c}ons\,: par l'am\'elioration de l'efficacit\'e globale (PTF) ou par l'accumulation de capital par unit\'e de production ($K/Y$). La d\'ecomposition exacte est\,:
\begin{equation}
    \Delta \ln\!\left(\frac{Y}{L}\right) = \frac{1}{1 - \bar\alpha}\,\Delta \ln A + \frac{\bar\alpha}{1 - \bar\alpha}\,\Delta \ln\!\left(\frac{K}{Y}\right),
    \label{eq:lp_decomp}
\end{equation}
o\`u $\bar\alpha$ est la part du capital dans la valeur ajout\'ee. Le premier terme mesure la contribution de la PTF, amplifi\'ee par le facteur $1/(1-\bar\alpha) > 1$. Le second mesure l'approfondissement du capital. Cette identit\'e ne requiert aucune forme fonctionnelle sp\'ecifique.

La figure~\ref{fig:labor_productivity} pr\'esente la productivit\'e du travail cumul\'ee depuis 1961. L'\'ecart entre la courbe observ\'ee et le contrefactuel sans approfondissement du capital mesure la contribution de $K/Y$. La PTF domine largement\,: c'est son effondrement apr\`es 2000 qui explique le ralentissement, et non un arr\^et de l'accumulation du capital.

\begin{figure}[H]
    \centering
    \includegraphics[width=\textwidth]{note_labor_productivity.png}
    \caption{La PTF explique l'essentiel de la croissance de la productivit\'e du travail --- le capital n'y contribue que modestement, 1961--2019.}
    \label{fig:labor_productivity}
\end{figure}

\begin{keyfinding}
\itshape Depuis 2000, la contribution de la PTF \`a la productivit\'e du travail est n\'egative ($-0.1$\,\% par ann\'ee), alors qu'elle atteignait $+1.7$\,\% avant 1980.
\end{keyfinding}

% ══════════════════════════════════════════════════════════════════════

\section*{L'effet Baumol}

La PTF agr\'eg\'ee est une moyenne pond\'er\'ee des PTF sectorielles, o\`u les poids sont les parts de valeur ajout\'ee nominale \citep{Hulten_1978}. Si les industries les plus productives voient leur part diminuer --- parce que leurs prix relatifs baissent --- la PTF agr\'eg\'ee cro\^it moins vite que ne le sugg\`ere la performance intra-sectorielle. C'est l'\emph{effet Baumol} \citep{Baumol_1967}.

La figure~\ref{fig:tfp_decomposition} d\'ecompose la PTF agr\'eg\'ee en une composante intra-sectorielle (structure \'economique constante) et un r\'esidu de composition. L'effet Baumol est visible mais modeste\,: l'essentiel de la variation de la PTF agr\'eg\'ee provient de ce qui se passe \textit{au sein} des industries.

\begin{figure}[H]
    \centering
    \includegraphics[width=\textwidth]{note_tfp_decomposition.png}
    \caption{L'effet Baumol p\`ese sur la PTF agr\'eg\'ee, mais la composante intra-sectorielle domine, 1961--2019.}
    \label{fig:tfp_decomposition}
\end{figure}

La figure~\ref{fig:baumol_scatter} illustre le m\'ecanisme\,: les industries \`a forte croissance de la PTF tendent \`a voir leur part dans la valeur ajout\'ee nominale diminuer. Cette corr\'elation n\'egative est la signature empirique de l'effet Baumol.

\begin{figure}[H]
    \centering
    \includegraphics[width=\textwidth]{note_baumol_scatter.png}
    \caption{Les industries \`a forte PTF perdent du poids dans le PIB nominal --- signature de l'effet Baumol, 1961--2019.}
    \label{fig:baumol_scatter}
\end{figure}

Le tableau~\ref{tab:note_decomposition} r\'esume les deux \'etapes. La croissance intra-sectorielle de la PTF est pass\'ee de 1.1\,\% par ann\'ee (1961--1980) \`a 0.1\,\% (2000--2019). L'effet Baumol est relativement stable autour de $-0.2$\,\% dans toutes les sous-p\'eriodes.

\includetables{note_decomposition}

\begin{keyfinding}
\itshape C'est la PTF \textit{au sein} des industries qui s'est effondr\'ee, et non un simple d\'eplacement de l'activit\'e vers des secteurs stagnants.
\end{keyfinding}

% ══════════════════════════════════════════════════════════════════════

\section*{Perspective internationale}

La m\'ethode pr\'esent\'ee est enti\`erement g\'en\'erique\,: elle ne d\'epend d'aucune hypoth\`ese sp\'ecifique au Canada. Aux \'Etats-Unis, le programme KLEMS du Bureau of Labor Statistics fournit des donn\'ees analogues. Pour les pays de l'OCDE, la base OECD STAN et le projet EU KLEMS offrent des comptes de productivit\'e sectoriels harmonis\'es.

% ══════════════════════════════════════════════════════════════════════

\bibliography{references}
\bibliographystyle{aer}

% ══════════════════════════════════════════════════════════════════════
%  APPENDIX
% ══════════════════════════════════════════════════════════════════════

\appendix
\renewcommand{\thesection}{\Alph{section}}

\section{M\'ethode}

\subsection{D\'erivation de la d\'ecomposition}

On part d'une fonction de production agr\'eg\'ee \`a rendements constants, $Y_t = A_t F(K_t, L_t)$, o\`u $F$ est homog\`ene de degr\'e 1. Par le th\'eor\`eme d'Euler et en d\'efinissant la part du capital $\alpha_t = F_K K_t / (F_K K_t + F_L L_t)$, on obtient en taux de croissance\,:
$\mathrm{d}\ln Y_t = \mathrm{d}\ln A_t + \alpha_t\,\mathrm{d}\ln K_t + (1-\alpha_t)\,\mathrm{d}\ln L_t$.
En soustrayant $\mathrm{d}\ln L_t$ et en r\'e\'ecrivant $\mathrm{d}\ln(K_t/L_t) = \mathrm{d}\ln(K_t/Y_t) + \mathrm{d}\ln(Y_t/L_t)$, on obtient l'\'equation~\eqref{eq:lp_decomp}.

\subsection{Agr\'egation par indices de T\"ornqvist}

Les agr\'egats sont construits par indices de T\"ornqvist \citep{Diewert_1976}. La PTF agr\'eg\'ee suit le th\'eor\`eme de \citet{Hulten_1978}\,: $\mathrm{d}\ln A_t = \sum_i \bar{S}_{it}\,\mathrm{d}\ln A_{it}$, o\`u $\bar{S}_{it}$ est la moyenne des parts de valeur ajout\'ee nominale du secteur $i$ en $t-1$ et $t$. Les agr\'egats de capital et de travail sont construits de mani\`ere analogue\,: $\mathrm{d}\ln K_t = \sum_i \bar\omega^K_{it}\,\mathrm{d}\ln K_{it}$ et $\mathrm{d}\ln L_t = \sum_i \bar\omega^L_{it}\,\mathrm{d}\ln L_{it}$, o\`u $\bar\omega^K_{it}$ et $\bar\omega^L_{it}$ sont les moyennes des parts de co\^ut du capital et de r\'emun\'eration du travail. La part agr\'eg\'ee du capital est $\alpha_t = \sum_i \text{co\^ut du capital}_{it} / \sum_i \text{VA}_{it}$, et $\bar\alpha_t = (\alpha_t + \alpha_{t-1})/2$.

\subsection{D\'ecomposition de la PTF agr\'eg\'ee}

En d\'ecomposant les poids courants $\bar{S}_{it}$ autour des poids d'une p\'eriode de r\'ef\'erence $S_{i,t_0}$\,:
\begin{equation*}
    \sum_{t=t_0}^{t_1} \mathrm{d}\ln A_t = \underbrace{\sum_{t}\sum_i S_{i,t_0}\,\mathrm{d}\ln A_{it}}_{\text{Intra-industries}} + \underbrace{\sum_{t}\sum_i (\bar{S}_{it} - S_{i,t_0})\,\mathrm{d}\ln A_{it}}_{\text{Composition (Baumol)}}.
\end{equation*}
Les parts de r\'ef\'erence sont r\'einitialis\'ees au d\'ebut de chaque sous-p\'eriode\,: $t_0 = 1961$ pour 1961--1980, $t_0 = 1980$ pour 1980--2000, et $t_0 = 2000$ pour 2000--2019.

Les donn\'ees proviennent du tableau 36-10-0217-01 de Statistique Canada, qui fournit des indices de PTF, de capital et de travail pour 38 industries SCIAN couvrant la p\'eriode 1961--2019. Les variables utilis\'ees sont\,: la PTF sectorielle, les indices de capital et de travail, la valeur ajout\'ee nominale, le co\^ut du capital et la r\'emun\'eration du travail. Le script \texttt{note.py} reproduit l'ensemble des r\'esultats.

\end{document}
