\documentclass[12pt]{article}
\usepackage{mathpazo, amsmath, amssymb, amsfonts, amsthm, microtype, tikz, graphicx, booktabs, caption, subcaption, bm, xfrac, appendix, setspace, comment, float, nth, fnpct, makecell, multirow, tabularx, threeparttable, titling, array, lipsum, mathtools, tocloft, tocvsec2}
\usepackage[margin=2.75cm]{geometry}
\renewcommand{\baselinestretch}{1.15}
\renewcommand{\arraystretch}{2}
\setlength\droptitle{-2.5cm}
%\setlength\parindent{0cm}
\setlength{\parskip}{0.15cm}
\definecolor{HEC}{rgb}{0, 0.1569, 0.3333}
\usepackage[colorlinks, allcolors=HEC]{hyperref}
\urlstyle{rm}
\captionsetup[figure]{labelfont={color=HEC}}
\captionsetup[table]{labelfont={color=HEC}}
\usepackage[ruled, vlined]{algorithm2e}
\setlength{\abovecaptionskip}{0.5cm}
\usepackage[longnamesfirst]{natbib}
\usepackage[french]{babel}

% Command to reference subfigures
\renewcommand{\thesubfigure}{(\alph{subfigure})}
\captionsetup[sub]{labelformat=simple}

% Define the path for the figures
\graphicspath{{../Figures/}}

% Define a command and the path for the tables
\newcommand*\includetables[1]{\input{../Tables/#1.tex}}

% Do not add sections of the main text to the appendix's table of content
\addtocontents{toc}{\protect\setcounter{tocdepth}{-1}}

% Command to comment within lines
\newcommand{\ignore}[1]{}

% Command to center table columns
\newcolumntype{C}[1]{>{\centering\arraybackslash}p{#1}}
\newlength\mcwidth

% Mathematical commands
\DeclareMathOperator*{\argmax}{arg\,max}
\newcommand{\diag}[1]{\text{diag}\left(#1\right)}
\newcommand{\partialof}[2]{\frac{\partial #1}{\partial #2}}
\newcommand{\totalof}[2]{\frac{\text{d} #1}{\text{d} #2}}
\newcommand{\deltaof}[3]{\delta #1 [#2; #3]}
\newcommand{\innerprod}[3]{\langle #1, #2 \rangle_{#3}}
\newtheorem{theorem}{Theorem}
\newtheorem{corollary}{Corollary}
\newtheorem{definition}{Definition}
\newtheorem{proposition}{Proposition}
\newtheorem{lemma}{Lemma}

% Command to resize equations
\newcommand{\resizeeq}[2]{\resizebox{#2\hsize}{!}{$#1$}}

% Command to make a small horizontal space
\newcommand{\smallquad}{\hspace{0.5em}}

% Title
\title{\textbf{Décomposition de la croissance de la productivité du travail agrégée }}
\author{\href{https://www.jeanfelixbrouillette.com}{Jean-F\'elix Brouillette}\textsuperscript{1}}
\date{\today}

\begin{document}

\maketitle

\footnotetext[1]{HEC Montr\'eal. E-mail: \href{mailto:jean-felix.brouillette@hec.ca}{jean-felix.brouillette@hec.ca}}

Ce document décrit une stratégie de décomposition progressive de la productivité du travail agrégée. L'approche repose sur des identités comptables simples et introduit graduellement des décompositions plus riches, mais toujours interprétables.

\paragraph{Étape 1: productivité du travail agrégée.} On part d'une fonction de production à rendements d'échelle constants $Y_t = A_t F(K_t, L_t)$, où $Y_t$ est le PIB réel, $K_t$ le capital, $L_t$ le travail et $A_t$ la PTF. Avec les parts factorielles dans le revenu observées, on obtient:
\begin{equation*}
    \Delta \ln\left(\frac{Y_t}{L_t}\right) = \frac{1}{1 - \alpha_t} \Delta \ln(A_t) + \frac{\alpha_t}{1 - \alpha_t} \Delta \ln\left(\frac{K_t}{Y_t}\right),
\end{equation*}
où $\alpha_t$ est la part du capital dans la valeur ajoutée. Cette décomposition ne requiert pas de forme fonctionnelle particulière. \textbf{Résultat attendu:} le ralentissement de la productivité du travail depuis les années 2000 s'explique presque entièrement par la PTF, non par l'accumulation du capital.

\paragraph{Étape 2: PTF agrégée.} On décompose la PTF agrégée en contributions sectorielles. Sous le théorème d'\citet{Hulten_1978}, la croissance de la PTF agrégée est une moyenne pondérée de la croissance des PTF sectorielles:
\begin{equation*}
    \mathrm{d}\ln(A_t) = \sum_{i=1}^N S_{it} \mathrm{d}\ln(A_{it}),
\end{equation*}
où $S_{it} \equiv P_{it} Y_{it} / P_t Y_t$ est la part du secteur $i$ dans le PIB nominal. En intégrant de $t_0$ à $t_1$:
\begin{equation*}
    \sum_{t=t_0}^{t_1} \mathrm{d}\ln(A_t) = \underbrace{\sum_{t=t_0}^{t_1} \sum_{i=1}^N S_{it_0} \mathrm{d}\ln(A_{it})}_{\text{Intra-industries}} + \underbrace{\sum_{t=t_0}^{t_1} \sum_{i=1}^N (S_{it} - S_{it_0}) \mathrm{d}\ln(A_{it})}_{\text{Inter-industries}}.
\end{equation*}
Le terme \textit{intra} mesure les gains d'efficacité au sein des secteurs; le terme \textit{inter} capture l'effet de la réallocation sectorielle.

\textbf{Implémentation.}
\begin{enumerate}
    \item Construire un panel sectoriel avec PTF et valeur ajoutée nominale.
    \item Calculer $S_{it}$ et $\Delta \ln(A_{it})$ par secteur-année.
    \item Appliquer les formules \textit{Intra} et \textit{Inter}.
    \item Agréger par sous-périodes.
\end{enumerate}

\textbf{Résultat attendu:} l'effet de composition (\textit{Inter}) est présent mais modeste; l'essentiel du ralentissement provient de la PTF intra-sectorielle.

\paragraph{Étape 3: PTF sectorielle.} On applique le théorème de Hulten au niveau des entreprises au sein d'une industrie $j$. Cette décomposition est une variante de \citet{Foster_Haltiwanger_Krizan_2001}, reformulée en taux de croissance pour assurer la cohérence avec l'Étape 2. Soit $s_{it}$ la part de l'entreprise $i$ dans le PIB nominal de l'industrie et $a_{it}$ sa PTF. La croissance de la PTF de l'industrie est:
\begin{equation*}
    \mathrm{d}\ln(A_{jt}) = \sum_{i} s_{it} \, \mathrm{d}\ln(a_{it}).
\end{equation*}
En distinguant les entreprises présentes aux deux dates ($C$), les entrantes ($E$) et les sortantes ($X$):
\begin{align*}
    \Delta \ln(A_{jt}) = \; &\underbrace{\sum_{i \in C} s_{i,t-1} \Delta \ln(a_{it})}_{\text{Intra-firmes}} + \underbrace{\sum_{i \in C} (s_{it} - s_{i,t-1}) \Delta \ln(a_{it})}_{\text{Inter-firmes}} \\[0.5em]
    &+ \underbrace{\sum_{i \in E} s_{it} (\ln a_{it} - \bar{a}_{j,t-1})}_{\text{Entrée}} - \underbrace{\sum_{i \in X} s_{i,t-1} (\ln a_{i,t-1} - \bar{a}_{j,t-1})}_{\text{Sortie}},
\end{align*}
où $\bar{a}_{jt} = \sum_i s_{it} \ln a_{it}$ est la PTF moyenne (en logarithmes) de l'industrie. L'entrée contribue positivement si les nouvelles entreprises sont plus productives que la moyenne initiale; la sortie contribue positivement si les sortantes étaient moins productives.

\textbf{PTFR.} La PTFR (productivité totale des facteurs revenu) est généralement la seule mesure disponible dans les microdonnées canadiennes. L'essentiel est d'assurer la cohérence entre la PTF et les parts $s_{it}$ (valeur ajoutée nominale).

\textbf{Intuition.}
\begin{itemize}
    \item \textit{Intra-firmes}: gains des entreprises en place.
    \item \textit{Inter-firmes}: réallocation vers les entreprises à forte croissance de la PTF.
    \item \textit{Entrée/Sortie}: productivité relative des firmes entrantes et sortantes.
\end{itemize}

\textbf{Implémentation.}
\begin{enumerate}
    \item Harmoniser les identifiants d'entreprises et la définition des industries.
    \item Construire la PTFR et les parts $s_{it}$ (valeur ajoutée nominale).
    \item Identifier entrants, sortants et firmes établies par paire d'années.
    \item Calculer $\Delta \ln(a_{it})$ pour les firmes établies; appliquer la décomposition.
    \item Agréger par industrie avec les parts de valeur ajoutée.
\end{enumerate}

Cette étape identifie si la faible croissance de la PTF provient des entreprises en place, d'une réallocation néfaste, ou des entrées et sorties d'entreprises.

\paragraph{Étape 4: Tendances intra-sectorielles.} Corréler les composantes de l'Étape 3 avec des mesures de: dynamisme d'affaires (taux d'entrée/sortie, croissance des jeunes entreprises), concentration de marché, investissements en capital et R\&D, etc.. Cette étape relie le diagnostic comptable à des mécanismes économiques et des pistes de politiques publiques.

\clearpage

\appendix

\paragraph{Dérivation de la décomposition agrégée.} On suppose $Y_t = A_t F(K_t, L_t)$ avec $F$ homogène de degré 1. Par le théorème d'Euler:
\begin{equation*}
    F(K_t, L_t) = F_K(K_t, L_t) K_t + F_L(K_t, L_t) L_t.
\end{equation*}
En taux de croissance, on obtient:
\begin{equation*}
    \mathrm{d}\ln(Y_t) = \mathrm{d}\ln(A_t) + \alpha_t \, \mathrm{d}\ln(K_t) + (1-\alpha_t)\mathrm{d}\ln(L_t),
\end{equation*}
où $\alpha_t \equiv \frac{F_K K_t}{F_K K_t + F_L L_t}$ correspond à la part du capital dans la valeur ajoutée (ou coût du capital dans les comptes). En soustrayant $\mathrm{d}\ln(L_t)$ des deux côtés:
\begin{equation*}
    \mathrm{d}\ln\left(\frac{Y_t}{L_t}\right) = \mathrm{d}\ln(A_t) + \alpha_t \, \mathrm{d}\ln\left(\frac{K_t}{L_t}\right).
\end{equation*}
En réécrivant $\mathrm{d}\ln(K_t/L_t)=\mathrm{d}\ln(K_t/Y_t) + \mathrm{d}\ln(Y_t/L_t)$ et en isolant $\mathrm{d}\ln(Y_t/L_t)$, on obtient la forme utilisée dans le texte:
\begin{equation*}
    \mathrm{d}\ln\left(\frac{Y_t}{L_t}\right) = \frac{1}{1-\alpha_t}\mathrm{d}\ln(A_t) + \frac{\alpha_t}{1-\alpha_t}\mathrm{d}\ln\left(\frac{K_t}{Y_t}\right).
\end{equation*}
Cette dérivation montre que la décomposition repose uniquement sur des rendements constants et sur l'utilisation des parts factorielles observées.

\clearpage

\bibliography{references}
\bibliographystyle{aer}

\end{document}
