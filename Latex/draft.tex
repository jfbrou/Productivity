\documentclass[12pt]{article}
\usepackage{mathpazo, amsmath, amssymb, amsfonts, amsthm, microtype, tikz, titling, graphicx, booktabs, caption, subcaption, bm, xfrac, appendix, setspace, comment, float, nth, fnpct, makecell, multirow, tabularx, threeparttable, titling, array, lipsum, mathtools, tocloft, tocvsec2}
\usepackage[margin=2.75cm]{geometry}
\renewcommand{\baselinestretch}{1.15}
\renewcommand{\arraystretch}{2}
\setlength\droptitle{-2.5cm}
%\setlength\parindent{0cm}
\setlength{\parskip}{0.15cm}
\definecolor{HEC}{rgb}{0, 0.1569, 0.3333}
\usepackage[colorlinks, allcolors=HEC]{hyperref}
\urlstyle{rm}
\captionsetup[figure]{labelfont={color=HEC}}
\captionsetup[table]{labelfont={color=HEC}}
\usepackage[ruled, vlined]{algorithm2e}
\setlength{\abovecaptionskip}{0.5cm}
\usepackage[longnamesfirst]{natbib}

% Command to reference subfigures
\renewcommand{\thesubfigure}{(\alph{subfigure})}
\captionsetup[sub]{labelformat=simple}

% Define the path for the figures
\graphicspath{{../Figures/}}

% Define a command and the path for the tables
\newcommand*\includetables[1]{\input{../Tables/#1.tex}}

% Do not add sections of the main text to the appendix's table of content
\addtocontents{toc}{\protect\setcounter{tocdepth}{-1}}

% Command to comment within lines
\newcommand{\ignore}[1]{}

% Command to center table columns
\newcolumntype{C}[1]{>{\centering\arraybackslash}p{#1}}
\newlength\mcwidth

% Mathematical commands
\DeclareMathOperator*{\argmax}{arg\,max}
\newcommand{\diag}[1]{\text{diag}\left(#1\right)}
\newcommand{\partialof}[2]{\frac{\partial #1}{\partial #2}}
\newcommand{\totalof}[2]{\frac{\text{d} #1}{\text{d} #2}}
\newcommand{\deltaof}[3]{\delta #1 [#2; #3]}
\newcommand{\innerprod}[3]{\langle #1, #2 \rangle_{#3}}
\newtheorem{theorem}{Theorem}
\newtheorem{corollary}{Corollary}
\newtheorem{definition}{Definition}
\newtheorem{proposition}{Proposition}
\newtheorem{lemma}{Lemma}

% Command to resize equations
\newcommand{\resizeeq}[2]{\resizebox{#2\hsize}{!}{$#1$}}

% Command to make a small horizontal space
\newcommand{\smallquad}{\hspace{0.5em}}

% Title
\title{\textbf{Where Did Canada's Productivity Growth Go?}\thanks{I am grateful to Edward Xu for excellent research assistance.}}
\author{\href{https://www.jeanfelixbrouillette.com}{Jean-F\'elix Brouillette}\textsuperscript{1}}
\date{\today}

\begin{document}

\maketitle

\footnotetext[1]{HEC Montr\'eal. E-mail: \href{mailto:jean-felix.brouillette@hec.ca}{jean-felix.brouillette@hec.ca}}

\begin{abstract}
    This paper quantifies the forces behind Canada's long-run productivity growth slowdown using an industry-level accounting framework for the period 1961--2019. I decompose aggregate total-factor productivity (TFP) growth into contributions from: (1) within-industry productivity improvements, (2) structural change (Baumol effects), and (3) factor reallocation across industries. The results show that Canada's post-2000 stagnation is driven almost entirely by a sharp decline in within-industry TFP growth. Structural change increasingly weighs on aggregate TFP as the economy shifts toward slower-growing service sectors, though its overall magnitude remains modest. Factor reallocation contributes little throughout the sample.
\end{abstract}

\clearpage

\section{Introduction}
\label{s:introduction}

Canada's productivity growth has slowed substantially over the past several decades, with the slowdown becoming especially pronounced at the turn of the millennium. This decline has translated directly into weaker wage growth and a noticeable erosion in the purchasing power of Canadian households. Understanding the origins of this slowdown requires disentangling whether it comes from declining productivity growth within industries or from structural shifts in economic activity across industries.

In this paper, I address this question using a growth-accounting framework that decomposes aggregate total-factor productivity (TFP) growth for the Canadian business sector into three components: within-industry productivity improvements, changes in the economy's industrial composition (Baumol effects), and the reallocation of capital and labor across industries. Building on the approach of \citet{Halperin_Mazlish_2024}, the decomposition isolates how much TFP growth can be attributed to technological and efficiency gains within industries and how much reflects shifts in value-added or inputs toward industries with faster or slower productivity growth.

Distinguishing these forces is critical because each carries different implications. If the slowdown reflects structural shifts toward sectors with low productivity growth (i.e., Baumol's cost disease), then weaker TFP growth may be an unavoidable consequence of preferences or technological constraints outside the reach of policy. If it reflects adverse patterns of factor reallocation, such as productive industries drawing too little capital or labor due to rising market power, the slowdown may instead reflect misallocation that competition policy could address. If the decline occurs within industries, it points toward underinvestment in technology adoption or development. Disentangling these forces can help us determine whether the Canadian TFP growth decline is a structural inevitability or a policy-relevant challenge.

I implement a growth-accounting decomposition using industry-level data (3-digit NAICS) on value-added, multifactor productivity, and capital and labor inputs from Statistics Canada for 1961–2019. In line with \citet{Baqaee_Farhi_2019}, I do not impose the efficiency conditions required for Hulten's theorem to hold, allowing factor reallocation to have first-order effects on aggregate productivity growth. The results show that Canada's post-2000 productivity stagnation is driven overwhelmingly by a sharp decline in within-industry TFP growth. Although Baumol effects increasingly weigh on aggregate productivity as activity shifts toward slow-growing sectors, their overall contribution remains modest, and factor reallocation plays only a minor role throughout the sample. These findings mirror the U.S. evidence documented by \citet{Halperin_Mazlish_2024}.

I also address the role of the resource sector, a frequent culprit in discussions of Canadian economic performance. I find that the oil and gas sector contributes disproportionately to the Baumol effect, accounting for approximately half of the structural drag. However, even after excluding the energy sector, the central finding holds: the ``pure'' productivity term (within-industry TFP growth) collapses from over 0.8\% per year prior to 2000 to just 0.2\% thereafter. This suggests that Canada's productivity growth malaise is not merely a symptom of its resource dependence or a mechanical shift to services, but a widespread failure to deliver efficiency gains across a broad range of industries.

The remainder of the paper is organized as follows. The following section reviews the relevant literature. Section \ref{s:theoretical framework} presents the theoretical framework underlying our productivity growth decomposition. Section \ref{s:data} describes the data used in the analysis. Section \ref{s:empirical results} presents the empirical results and Section \ref{s:data} discusses them. Section \ref{s:conclusion} concludes.

\subsection*{Literature review}

Our analysis is grounded in the foundational literature on productivity measurement. \citet{Hulten_1978}, building on the work of \citet{Solow_1957} and \citet{Domar_1961}, showed that in efficient economies, aggregate TFP growth can be expressed as a sales-weighted average of industry-level productivity growth. More recent work relaxed the assumptions underlying Hulten's theorem. Notably, \citet{Baqaee_Farhi_2019} show that factor reallocation can have first-order effects on aggregate TFP in distorted economies. I adopt their framework to decompose Canadian TFP growth into three distinct components: within-industry productivity improvements, Baumol effects, and factor reallocation.

This framework allows us to directly engage with the literature on structural change and ``Baumol's cost disease.'' \citet{Baumol_1967} famously theorized that as technologically progressive sectors drive up economy-wide wages, stagnant sectors--unable to offset rising labor costs with efficiency gains--experience increasing relative prices. If consumer demand for these stagnant services is price inelastic, they capture an ever-growing share of nominal GDP, mechanically dampening aggregate productivity growth. \citet{Nordhaus_2006} and \citet{Halperin_Mazlish_2024} took this insight to the data by measuring sectoral contributions to U.S. productivity growth and highlighting the central role of this sectoral reallocation. \citet{Duernecker_Herrendorf_Valentinyi_2023} develop a macroeconomic model of structural change and argue that Baumol's cost disease in the U.S. will be less severe in the future than it was in the past. 

Within the Canadian context, a growing body of work documents and examines the productivity problem. A long tradition of empirical work documents Canada's persistent productivity underperformance relative to the United States. Early analyses--such as \citet{Dion_2007}, \citet{Boothe_Roy_2008}, \citet{Baldwin_Gu_2009}, and \citet{Sharpe_2010}--highlight chronically weak multifactor productivity growth and an emerging pattern of slower technical progress. More recent contributions bring additional structure to the diagnosis. \citet{Conesa_Pujolas_2019} show that the 2002--2014 stagnation cannot be explained by measurement issues or simple compositional effects. \citet{Loertscher_Pujolas_2024} show that the oil and gas extraction sector explains much of Canada's relative productivity growth gap vis-à-vis the United States. Complementing this evidence, \citet{Alexopoulos_Cohen_2018} and \citet{Mollins_St-Amant_2019} document a slowdown in technology commercialization and ICT adoption, suggesting these phenomena have contributed directly to weaker TFP growth. Finally, \citet{Kiarsi_2025} decomposes Canada's real GDP per capita growth and confirms that the deceleration in living standards is overwhelmingly driven by the TFP growth slowdown.

Our contribution is to place these different explanations within a unified growth-accounting framework capable of quantifying the relative importance of within-industry technical progress, structural change, and allocative efficiency over six decades of Canadian economic history. In doing so, I extend the existing evidence in three directions: (1) I examine a long time period (1961--2019), (2) I incorporate allocative inefficiencies using the \citet{Baqaee_Farhi_2019} theoretical framework, and (3) I quantify the contribution of sectoral reallocation relative to the collapse in within-industry productivity growth.

\section{Theoretical framework}
\label{s:theoretical framework}

To decompose the sources of Canadian productivity growth, I adopt the general equilibrium growth-accounting framework of \citet{Baqaee_Farhi_2019}. This approach allows us to aggregate industry-level shocks into economy-wide TFP growth while explicitly accounting for input-output linkages and allocative inefficiencies.

\paragraph{Input-output definitions.} Consider an economy with $N$ industries indexed by $i \in \{1, \ldots, N\}$ and two factors of production: capital and labor. Let $\mathbf{b}_t$ be the $(N + 2) \times 1$ vector whose $i$-th element is equal to the share of industry $i$ in aggregate nominal value-added:
\begin{equation*}
    b_{it} \equiv \frac{P_{it} Y_{it}}{\sum_{j=1}^N P_{jt} Y_{jt}}, \quad \forall i \{1, \ldots, N\}.
\end{equation*}
The first $N$ elements of $\mathbf{b}_t$ correspond to industries, while the last two correspond to capital and labor. Since factors do not enter in final demand, the last two elements of $\mathbf{b}_t$ are equal to zero. Let us further define the cost-based input-output matrix $\bm{\Omega}_t$ of dimension $(N + 2) \times (N + 2)$ whose $(i, j)$-th element is equal to industry $i$'s expenditures on inputs from $j$ as a share of its total expenditures:
\begin{equation*}
    \Omega_{ijt} \equiv \frac{P_{jt} X_{ijt}}{\sum_{k=1}^N P_{kt} X_{ikt}}.
\end{equation*}
The first $N$ rows of $\bm{\Omega}_t$ correspond to industries, while the last two correspond to capital and labor. Since capital and labor require no intermediate inputs, the last two rows of $\bm{\Omega}_t$ are filled with zeros. The Leontief inverse of the cost-based input-output matrix is defined as:
\begin{equation*}
    \bm{\Psi}_t \equiv (\mathbf{I} - \bm{\Omega}_t)^{-1}.
\end{equation*}
Finally, the cost-based Domar weights are defined as:
\begin{equation*}
    \bm{\lambda}_t^{\prime} \equiv \mathbf{b}_t^{\prime} \bm{\Psi}_t.
\end{equation*}
For expositional convenience, we denote by $\lambda_t^K$ and $\lambda_t^L$ the last two elements of $\bm{\lambda}_t^{\prime}$, which measure their importance in final demand, indirectly through the production network of the economy.

\paragraph{Total-factor productivity.} As shown by \citet{Baqaee_Farhi_2019}, aggregate TFP growth in inefficient economies can be calculated as:
\begin{equation*}
    \mathrm{d}\ln(A_t) = \mathrm{d}\ln(Y_t) - \lambda_t^K \mathrm{d}\ln(K_t) - \lambda_t^L \mathrm{d}\ln(L_t)
\end{equation*}
where $A_t$ is aggregate TFP, $Y_t$ is aggregate real value-added, $K_t$ is aggregate capital, and $L_t$ is aggregate labor. The growth rate of aggregate real value-added is given by:
\begin{equation*}
    \mathrm{d}\ln(Y_t) \equiv \sum_{i=1}^N b_{it} \mathrm{d}\ln(Y_{it}).
\end{equation*}
Here, the growth rate of real value-added in industry $i$ is given by:
\begin{equation*}
    \mathrm{d}\ln(Y_{it}) = \mathrm{d}\ln(A_{it}) + \alpha_{it}^K \mathrm{d}\ln(K_{it}) + \alpha_{it}^L \mathrm{d}\ln(L_{it})
\end{equation*}
where $A_{it}$ is industry $i$'s TFP, $K_{it}$ its capital input, $L_{it}$ its labor input, and $\alpha_{it}^K$ and $\alpha_{it}^L$ are the elasticities of its output with respect to capital and labor, respectively. Here, we make the standard assumption of constant returns to scale at the industry level such that $\alpha_{it}^K + \alpha_{it}^L = 1$. Under the additional assumption that the representative firm in industry $i$ minimizes its costs and is a price-taker in input markets, the elasticity of output with respect to the capital input is given by:
\begin{equation*}
    \alpha_{it}^K = \frac{r_{it} K_{it}}{r_{it} K_{it} + w_{it} L_{it}}.
\end{equation*}
The growth rate of aggregate capital is given by:
\begin{equation*}
    \mathrm{d}\ln(K_t) \equiv \sum_{i=1}^N \omega_{it}^K \mathrm{d}\ln(K_{it}) \quad \text{where} \quad \omega_{it}^K \equiv \frac{r_{it} K_{it}}{\sum_{j=1}^N r_{jt} K_{jt}}
\end{equation*}
and the growth rate of aggregate labor is given by:
\begin{equation*}
    \mathrm{d}\ln(L_t) \equiv \sum_{i=1}^N \omega_{it}^L \mathrm{d}\ln(L_{it}) \quad \text{where} \quad \omega_{it}^L \equiv \frac{w_{it} L_{it}}{\sum_{j=1}^N w_{jt} L_{jt}}
\end{equation*}
and where $r_{it}$ and $w_{it}$ are the rental rate of capital and the wage rate in industry $i$ at time $t$, respectively.

\paragraph{Productivity growth accounting.} Combining the above equations, we can express aggregate TFP growth as:
\begin{align*}
    \mathrm{d}\ln(A_t) &= \sum_{i=1}^N b_{it} \mathrm{d}\ln(A_{it}) \\
    &+ \sum_{i=1}^N (b_{it} \alpha_{it}^K - \omega_{it}^K \lambda_t^K) \mathrm{d}\ln(K_{it}) \\
    &+ \sum_{i=1}^N (b_{it} \alpha_{it}^L - \omega_{it}^L \lambda_t^L) \mathrm{d}\ln(L_{it}).
\end{align*}
Following \citet{Halperin_Mazlish_2024}, adding and subtracting the term $\sum_{i=1}^N b_{it_0} \mathrm{d}\ln(A_{it})$ and summing from time $t_0$ to $t_1$, we can further decompose the cumulative aggregate TFP growth between these two dates into the following components:
\begin{alignat}{2}
    \sum_{t=t_0}^{t_1} \mathrm{d}\ln(A_t) &= \sum_{t=t_0}^{t_1} \sum_{i=1}^N b_{it_0} \mathrm{d}\ln(A_{it}) &&\quad \text{\textcolor{HEC}{Productivity}} \label{eq:productivity_1} \\
    &+ \sum_{t=t_0}^{t_1} \sum_{i=1}^N (b_{it} - b_{it_0}) \mathrm{d}\ln(A_{it}) &&\quad \text{\textcolor{HEC}{Baumol}} \label{eq:baumol_1} \\
    &+ \sum_{t=t_0}^{t_1} \sum_{i=1}^N (b_{it} \alpha_{it}^K - \omega_{it}^K \lambda_t^K) \mathrm{d}\ln(K_{it}) &&\quad \text{\textcolor{HEC}{Capital}} \label{eq:capital} \\
    &+ \sum_{t=t_0}^{t_1} \sum_{i=1}^N (b_{it} \alpha_{it}^L - \omega_{it}^L \lambda_t^L) \mathrm{d}\ln(L_{it}) &&\quad \text{\textcolor{HEC}{Labor.}} \label{eq:labor}
\end{alignat}
The interpretation of these four terms is as follows.

\paragraph{1. Within-industry productivity.} Term \eqref{eq:productivity_1} weighs industry-level TFP growth by each industry's initial share of value-added. That is, if we freeze each industry's share of nominal value-added, we capture the contribution of within-industry productivity improvements alone--how much aggregate TFP has grown purely because each industry became more productive over time.

\paragraph{2. The Baumol effect.} Term \eqref{eq:baumol_1} instead weighs industry-level TFP growth by the change in each industry's share of value-added between dates $t_0$ and $t_1$. Therefore, it measures how the changing industrial composition of the economy contributes to aggregate TFP growth. This term is negative when industries with low productivity growth become more important in the economy, or vice-versa.

\paragraph{3. Factor reallocation.} Terms \eqref{eq:capital} and \eqref{eq:labor} capture the contribution of reallocating factors across sectors. It weighs changes in industry-level capital and labor inputs by the difference between two terms. The first term is the product of the industry's share of total value-added and the elasticity of its output with respect to inputs. The second term is the product of the industry's share of total factor expenditures and the elasticity of aggregate output with respect to inputs. Intuitively, moving inputs into industries where they are more productive than the economy-wide average ($\alpha_{it}^K > \lambda_t^K$ or $\alpha_{it}^L > \lambda_t^L$) and where the receiving industries are more important in output than costs ($b_{it} > \omega_{it}^K$ or $b_{it} > \omega_{it}^L$) (i.e., high markup industries) raises aggregate productivity growth.

\section{Data}
\label{s:data}

Our empirical analysis covers the Canadian economy over the period 1961 to 2019. To implement the decomposition described in Section \ref{s:theoretical framework}, I construct a dataset combining industry-level productivity measures with input-output linkages at the 3-digit North American Industry Classification System (NAICS) level. The data are drawn from two primary Statistics Canada sources.

\paragraph{Productivity and factor inputs.} I use industry-level production data from Statistics Canada Table 36-10-0217-01 (Multifactor productivity, value-added, capital input and labour input in the aggregate business sector and major sub-sectors). This table provides the core variables required for the growth accounting components of our decomposition:
\begin{itemize}
    \item \textbf{Output and TFP:} Nominal value-added ($P_{it} Y_{it}$) is measured using Gross Domestic Product at basic prices. For technical change, I use the index of Multifactor Productivity ($A_{it}$) based on value-added.
    \item \textbf{Factor inputs:} I obtain indices for capital input ($K_{it}$) and labor input ($L_{it}$) to track real factor growth.
    \item \textbf{Factor weights:} To construct the required cost shares and elasticities, I use data on capital cost ($r_{it} K_{it}$) and labor compensation ($w_{it} L_{it}$).
\end{itemize}

\paragraph{Input-output linkages.} To construct the cost-based Domar weights, I use the Symmetric Input-Output Tables from Statistics Canada Table 36-10-0001-01 and its historical predecessors. These tables provide the intermediate input expenditures ($\{P_{jt} X_{ijt}\}_{j=1}^N$) between industries. I use these flows to construct the cost-based input-output matrix $\bm{\Omega}_t$ and the resulting Leontief inverse $\bm{\Psi}_t$ for each year in our sample. By harmonizing these two data sources at the 3-digit NAICS level, we obtain a balanced panel of $N$ industries spanning nearly six decades, allowing us to track the evolution of aggregate TFP through the lens of the \citet{Baqaee_Farhi_2019} framework.

\section{Empirical results}
\label{s:empirical results}

The findings are presented in three steps. First, I examine the raw correlations in the industry-level data to test for the ``symptoms'' of Baumol's cost disease as in \citet{Nordhaus_2006}. Second, I present the decomposition of aggregate TFP growth to isolate the relative contributions of within-industry growth, structural change, and reallocation. Third, I analyze the sectoral contributions, with a specific focus on the outsized role of the oil and gas sector.

\subsection{The symptoms of a cost disease}

Before turning to the formal decomposition, I first assess whether Canadian data display the classic features of unbalanced growth identified by \citet{Baumol_1967}. Baumol's hypothesis rests on a particular transmission mechanism: in industries with rapid productivity growth, efficiency improvements push relative prices downward and wages upward. Stagnant industries, drawing on the same labor pool, must match these wage increases despite lacking corresponding productivity gains. To preserve profit margins under rising labor costs, firms in these sectors are compelled to raise prices. If consumer demand for their services is price inelastic, households continue to purchase them even as costs climb. As a result, these stagnant, high-cost sectors absorb an increasingly large share of nominal GDP.

Figure \ref{fig:price_wage_tfp_growth} offers strong evidence for the first half of this mechanism. Indeed, Panel \ref{fig:price_tfp_growth} documents a pronounced negative relationship between long-run relative price growth and TFP growth across Canadian industries: sectors with rapid productivity gains (e.g., computer and electronic product manufacturing or wood product manufacturing) have experienced substantial price declines, while more stagnant sectors (e.g., arts, entertainment, and recreation or oil and gas extraction) have undergone marked price inflation. By contrast, Panel \ref{fig:wage_tfp_growth} shows no systematic association between wage growth and productivity growth, in line with Baumol's hypothesis.

\begin{figure}[h!]
    \centering
    \caption{Prices, wages, and TFP growth}
    \begin{subfigure}[b]{0.48\textwidth}
        \centering
        \caption{Prices}
        \includegraphics[width=\textwidth]{price_tfp_growth.png}
        \label{fig:price_tfp_growth}
    \end{subfigure}
    \hfill
    \begin{subfigure}[b]{0.48\textwidth}
        \centering
        \caption{Wages}
        \includegraphics[width=\textwidth]{wage_tfp_growth.png}
        \label{fig:wage_tfp_growth}
    \end{subfigure}
    \label{fig:price_wage_tfp_growth}
\end{figure}

Figure \ref{fig:gdp_tfp_growth} corroborates the second part of the mechanism: the ``stagnant'' sectors are indeed capturing a larger share of the economy. Panel \ref{fig:va_tfp_growth} shows a pronounced negative relationship between TFP growth and nominal GDP growth. As technologically progressive manufacturing industries contract in relative value because of declining prices, spending shifts toward low-productivity service sectors. Panel \ref{fig:real_va_tfp_growth} reveals a far weaker correlation between real GDP growth and TFP growth, indicating that demand for stagnant sectors is indeed relatively inelastic. Taken together, these patterns suggest that the Canadian economy exhibits the key symptoms of Baumol's cost disease.

\begin{figure}[h!]
    \centering
    \caption{GDP and TFP growth}
    \begin{subfigure}[b]{0.48\textwidth}
        \centering
        \caption{Nominal GDP}
        \includegraphics[width=\textwidth]{va_tfp_growth.png}
        \label{fig:va_tfp_growth}
    \end{subfigure}
    \hfill
    \begin{subfigure}[b]{0.48\textwidth}
        \centering
        \caption{Real GDP}
        \includegraphics[width=\textwidth]{real_va_tfp_growth.png}
        \label{fig:real_va_tfp_growth}
    \end{subfigure}
    \label{fig:gdp_tfp_growth}
\end{figure}

In Appendix \ref{a:appendix}, I show that these patterns are remarkably stable over time. Replicating Figures \ref{fig:price_wage_tfp_growth} and \ref{fig:gdp_tfp_growth} for the periods 1961--1980, 1980--2000, and 2000--2019 reveals that the same relationships persist across all three intervals. This robustness suggests that the forces underlying Baumol's cost disease are not episodic but structural: for decades, sectors with slow productivity growth have consistently seen their prices rise and their market share expand. I also show that input (labor and capital) cost shares are not systematically related to TFP growth.

\subsection{The productivity growth diagnosis}

To quantify these forces, Table \ref{tab:tfp_decomposition_1} decomposes aggregate TFP growth into its three constitutive parts: within-industry productivity (\textit{Productivity}), structural change (\textit{Baumol}), and factor reallocation (\textit{Capital} and \textit{Labor}). The table reports this decomposition for the full sample (1961--2019) as well as for three sub-intervals: 1961--1980, 1980--2000, and 2000--2019.

\includetables{tfp_decomposition_1}

Across the entire sample, aggregate TFP growth averages 0.5\% per year and is overwhelmingly propelled by within-industry productivity gains (0.78 percentage points). The Baumol component contributes –0.27 percentage points, indicating that the economy's gradual shift toward slow-productivity sectors exerts a sizable drag on aggregate growth. Factor reallocation plays only a minor role: capital and labor reallocations contribute –0.11 and 0.10 percentage points, respectively, largely offsetting one another. Figure \ref{fig:tfp_decomposition_1} visually summarizes these results.

\begin{figure}[h!]
    \centering
    \caption{TFP growth decomposition}
    \includegraphics[width=0.8\textwidth]{tfp_decomposition_1.png}
    \label{fig:tfp_decomposition_1}
\end{figure}

However, the full-sample decomposition masks a striking shift over time. In the early period (1961–1980), aggregate TFP growth averages nearly 1\% per year, but by 2000–2019 it falls to –0.1\% per year. The decomposition highlights three central insights behind this TFP growth collapse.

\paragraph{The collapse occurs within industries.} The first row of Table \ref{tab:tfp_decomposition_1} shows that the slowdown is clearly a within-industry phenomenon. From 1961 to 1980, within-sector productivity gains contributed 1.11\% to annual TFP growth; by 1980--2000, this contribution had slowed to 0.76\%, and after 2000 it fell sharply to just 0.12\%, explaining most of the decline. These patterns indicate that the core issue is not a structural reallocation toward low-productivity sectors but a widespread loss of momentum in productivity improvements within industries.

\paragraph{The Baumol effect is a steady headwind.} The Baumol term is persistently negative, confirming that economic activity continues to shift toward slower-growing sectors. Its magnitude, however, evolves only modestly--from –0.08\% in 1961--1980 to –0.16\% in 1980--2000 and –0.20\% in 2000--2019. Although this structural drag aggravates the slowdown, it is not its principal driver. Figure \ref{fig:baumol} illustrates the cumulative impact of this drag: absent this adverse reallocation, aggregate Canadian TFP would be roughly 12\% higher by 2019.

\begin{figure}[h!]
    \centering
    \caption{TFP growth and the Baumol effect}
    \includegraphics[width=0.8\textwidth]{baumol.png}
    \label{fig:baumol}
\end{figure}

\paragraph{Factor reallocation plays a minor role.} The contributions from capital and labor reallocation are small and largely offset each other, indicating that the slowdown cannot be attributed to a worsening of allocative inefficiency that would impede resources from flowing toward high-productivity sectors. This is noteworthy in light of growing concerns that market power has been increasing in Canada, potentially distorting competitive pressures and resource allocation \citep{CBC_2023}. The decomposition suggests that, despite these concerns, factor misallocation is not the primary force behind the aggregate TFP growth decline.

\subsection{The sectoral anatomy of the slowdown}

While our aggregate decomposition reveals that the Baumol effect is not the dominant driver of Canada's productivity stagnation, it remains highly informative to examine the specific industry-level forces at play. Our theoretical framework allows us to additively separate the contribution of each industry to aggregate TFP growth, providing a granular look at where efficiency is being gained or lost. Figure \ref{fig:tfp_contribution_industry} presents this industrial breakdown over the entire sample period. The results highlight a stark contrast across sectors: while retail and certain manufacturing industries have contributed positively to aggregate TFP growth, the oil and gas extraction, FIRE (finance, insurance, and real estate), and mining sectors have exerted significant negative pressures.

\begin{figure}[h!]
    \centering
    \caption{Industrial contributions to TFP growth}
    \includegraphics[width=0.9\textwidth]{tfp_contribution_industry.png}
    \label{fig:tfp_contribution_industry}
\end{figure}

The sources of these negative contributions become clearer when comparing industry-level TFP growth (Figure \ref{fig:tfp_growth_industry}) with nominal GDP growth (Figure \ref{fig:va_growth_industry}) in Appendix \ref{a:appendix}. The sectors with the largest negative contributions--especially oil and gas extraction--are those experiencing the deepest TFP contractions in the entire dataset. Yet these same industries have expanded markedly in nominal terms. This decoupling, in which a sector becomes less efficient while claiming a larger share of the economy, is the defining signature of the Baumol-style structural drag highlighted earlier.

Given this particularly large footprint of resource industries, it is natural to ask whether Canada's productivity malaise is essentially a resource-sector phenomenon. To assess this, Table \ref{tab:tfp_decomposition_1_no_oge} replicates our decomposition after excluding oil and gas extraction. The results show that the resource sector does indeed amplify the slowdown: removing it raises post-2000 aggregate TFP growth from –0.09\% to a modestly positive 0.11\%. However, the broader picture remains unchanged. Even after excluding oil and gas, within-industry productivity growth collapses from 0.82\% to just 0.21\%. Thus, while resource-specific headwinds are an important part of the story, they are layered atop a deeper, economy-wide decline in within-industry efficiency.

\includetables{tfp_decomposition_1_no_oge}

% ZZZ: talk about the role of the mining and FIRE sectors.

\subsection{A comparison with the United States}
\label{s:usa}

To place the Canadian experience in perspective, I compare our results with those for the United States. \citet{Halperin_Mazlish_2024} apply the same decomposition to U.S. data over the 1947--2016 period to analyze the post-1973 productivity slowdown, which is presented in Table \ref{tab:tfp_decomposition_usa}. The parallels between the two economies are striking.

\begin{table}[h!]
    \centering
    \begin{threeparttable}
        \caption{U.S. TFP growth decomposition}
        \begin{tabular}{lcccc}
        \hline
        \hline
        & & 1947--2016 & Pre-1973 & Post-1973 \\
        \hline
        Productivity & & 1.04\% & 1.52\% & 0.77\% \\
        Baumol & & -0.28\% & -0.13\% & -0.37\% \\
        Capital & & -0.01\% & -0.04\% & -0.01\% \\
        Labor & & -0.00\% & -0.00\% & -0.01\% \\
        \hline
        Total & & 0.75\% & 1.35\% & 0.41\% \\
        \hline
        \hline
        \end{tabular}
        \begin{tablenotes}[flushleft]
        \footnotesize
        \item \textit{Note:} This table reproduces the U.S. TFP growth decomposition from \citet{Halperin_Mazlish_2024} using their reported estimates. The sample covers the period 1947--2016, with a break at 1973 to capture the post-oil shock productivity slowdown.
        \end{tablenotes}
        \label{tab:tfp_decomposition_usa}
    \end{threeparttable}
\end{table}

First, the dominant source of stagnation is the same on both sides of the border. \citet{Halperin_Mazlish_2024} show that the U.S. productivity slowdown is driven overwhelmingly by a collapse in the within-industry component; they estimate that this contribution fell by roughly 0.75 percentage points after 1973. This mirrors our results for Canada and suggests that the slowdown is not an idiosyncratic national phenomenon but part of a broader global deceleration in technological progress--a pattern consistent with the view that ``ideas are getting harder to find'' \citep{Bloom_Jones_VanReenen_Webb_2020}.

Second, both countries experience a persistent yet secondary drag from structural change. \citet{Halperin_Mazlish_2024} estimate that Baumol's cost disease accounts for around one-quarter of the U.S. slowdown, subtracting about 25 basis points annually. Our Canadian decomposition yields a remarkably similar pattern: the Baumol term reduces growth by about 0.27 percentage points per year over the full sample. In each case, the shift toward stagnant service sectors acts as a steady headwind, not the principal driver of the post-2000 decline. Finally, the evidence on factor reallocation is broadly comparable, suggesting that static misallocation is unlikely to be the primary force behind North America's productivity slump.

% ZZZ: talk about the cross-border Baumol cost disease.

% ZZZ: talk about the growth vs. level differences.

\section{Discussion}
\label{s:discussion}

Our growth-accounting decomposition yields a clear conclusion: the productivity growth slowdown in Canada is driven by a decline in within-industry efficiency improvements. The direct contributions of structural change (Baumol effects) and factor misallocation appear modest by comparison. However, interpreting these components as entirely independent forces may be misleading. It is plausible that the macroeconomic forces driving structural change and factor reallocation exert significant indirect influence on within-industry innovation and technology adoption incentives.

Consider the role of the resource sector. While our decomposition isolates the direct ``Baumol drag'' of this sector, it does not capture the general equilibrium consequences of a resource boom on other industries. If the Canadian economy suffers from a ``Dutch disease''--where strong resource exports drive up the real exchange rate--technologically progressive tradeable sectors (such as manufacturing) may become less competitive in international markets. This contraction in their effective market size may thus reduce the return on investments in research and development (R\&D), consistent with the findings of \citet{Aghion_Bergeaud_Lequien_Melitz_2024}. In this scenario, the resource boom would manifest in our accounting framework not just as a Baumol effect, but as a collapse in the ``within-industry'' TFP growth contribution of manufacturing industries, as their incentives to innovate are eroded by currency appreciation.

There is evidence supporting this channel in the Canadian context. Indeed, \citet{Baldwin_Yan_2015} show that larger export markets enhance productivity by enabling Canadian firms to exploit economies of scale and product specialization, intensifying competitive pressure that rewards efficiency, and expanding the incentives and opportunities for firms to innovate. Complementarily, \citet{Baldwin_Gu_Yan_2013} link Canada's manufacturing productivity slowdown to declining export growth and rising excess capacity, consistent with exchange-rate pressures during the 2000s commodity boom.

A similar logic applies to the role of market power. The reallocation terms in our decomposition only capture the static cost of resource misallocation--the efficiency lost because capital and labor are not directed towards the industries where their marginal product is highest. They do not, however, capture the dynamic consequences of market power. R\&D typically entails high fixed costs; therefore, the incentive to invest in technology adoption or productivity improvements is increasing in the scale of production. If rising market power leads firms to restrict output to support prices, it simultaneously reduces the scale over which they can amortize the costs of innovation \citep{Arrow_1962}. Consequently, distortions that appear quantitatively small in the static reallocation terms may nonetheless be responsible for a significant portion of the decline in the within-industry residual by depressing firm-level incentives to develop or adopt new technologies.

%This perspective aligns closely with a broader Canadian literature documenting a gradual erosion of competitive intensity. Studies such as \citet{Sharpe_2008} and the Industry Canada review summarized in \citet{Roy_2008} argue that weaker competition helps explain part of Canada’s long-standing productivity gap relative to the United States. More recent evidence from the Competition Bureau \citep{CompetitionBureau_2023} points to rising concentration, declining firm entry, and increasing markups across several industries since 2000—patterns consistent with diminished competitive pressure and weaker incentives for firms to innovate. Taken together, these findings reinforce the idea that forces not prominently visible in our static reallocation terms may nonetheless operate through dynamic channels that depress within-industry productivity growth, amplifying the aggregate slowdown our decomposition uncovers.

Therefore, while our results highlight the \textit{proximate} cause of the slowdown as a failure of within-industry efficiency, the \textit{ultimate} causes may still be rooted in the structural composition of the economy and the competitive environment. Policy interventions addressing market power, misallocation, or sectoral imbalances may thus yield productivity gains that exceed the modest direct contributions inferred in our static decomposition.

\section{Conclusion}
\label{s:conclusion}

In this paper, I revisited the puzzle of Canada's productivity slowdown using a general growth-accounting framework that accounts for input-output linkages, allocative inefficiencies, and structural change. By decomposing aggregate TFP growth from 1961 to 2019, I sought to determine whether Canada's stagnation reflects an inevitable structural shift toward services (Baumol's cost disease), a worsening allocation of resources, or a widespread decline in technical efficiency growth.

Our results point to a clear diagnosis: the post-2000 TFP growth stagnation is driven overwhelmingly by a collapse in within-industry productivity growth. While the contribution of ``pure'' technical progress averaged 0.76\% annually between 1980 and 2000, it fell to just 0.12\% in the subsequent two decades. By contrast, while I find evidence of a persistent ``Baumol cost disease''--whereby economic activity shifts toward stagnant, high-price sectors--this force acts as a stable headwind rather than the driver of the recent deceleration. Similarly, the direct contribution of factor reallocation remains quantitatively small throughout the sample.

I also scrutinized the role of the resource sector, often cited as a source of Canadian economic underperformance \citep{Loertscher_Pujolas_2024}. I find that the oil and gas industry is indeed an important source of structural drag, accounting for approximately half of the economy-wide Baumol effect. However, the core finding remains robust to excluding the energy sector: even in the non-resource economy, within-industry productivity growth has collapsed. This suggests that the productivity crisis is not a localized symptom of the resource boom, but a broad-based malaise.

These results carry important policy implications. If the slowdown were primarily a misallocation problem, competition policy and barriers to factor mobility would be the natural levers. If it were purely a Baumol effect, the slowdown might be regarded as a structural inevitability. Instead, the dominance of the within-industry component points to weakened incentives for innovation, technology adoption, and productivity- or quality-enhancing investments. As discussed, forces such as misallocation, reduced competitive intensity, and exchange-rate dynamics may not leave strong signatures in our decomposition's static reallocation terms, yet they may still suppress productivity growth through dynamic channels.

Ultimately, our findings underscore the need for a deeper investigation into the microeconomic origins of the within-industry productivity collapse. Aggregate decompositions can reveal \textit{where} growth is faltering, but they cannot identify the firm-level mechanisms behind the slowdown. Progress on this front requires micro-datasets that track firms with sufficient granularity to disentangle different possible explanations. By opening this microeconomic black box, future work can move from diagnosing the slowdown to understanding--and ultimately addressing--the structural forces that have eroded Canadian productivity growth.

\clearpage

\bibliography{references}
\bibliographystyle{aer}

\clearpage
\appendix
\section{Appendix}
\label{a:appendix}
\numberwithin{equation}{section}
\renewcommand\thefigure{\thesection.\arabic{figure}}
\renewcommand\thetable{\thesection.\arabic{table}}

\begin{comment}
To calculate and decompose aggregate TFP growth in Canada, we need data on the following variables:
\begin{enumerate}
    \item Industry-level nominal value-added ($\{P_{it} Y_{it}\}_{i=1}^N$): We use data from 1961 to 2019 on gross domestic product at the 3-digit NAICS level from \href{https://www150.statcan.gc.ca/t1/tbl1/en/tv.action?pid=3610021701}{Table 36-10-0217-01} of Statistics Canada.
    \item Industry-level total-factor productivity ($\{A_{it}\}_{i=1}^N$): We use data from 1961 to 2019 on multifactor productivity based on value-added at the 3-digit NAICS level from \href{https://www150.statcan.gc.ca/t1/tbl1/en/tv.action?pid=3610021701}{Table 36-10-0217-01} of Statistics Canada.
    \item Industry-level capital ($\{K_{it}\}_{i=1}^N$): We use data from 1961 to 2019 on capital inputs at the 3-digit NAICS level from \href{https://www150.statcan.gc.ca/t1/tbl1/en/tv.action?pid=3610021701}{Table 36-10-0217-01} of Statistics Canada.
    \item Industry-level labor ($\{L_{it}\}_{i=1}^N$): We use data from 1961 to 2019 on labor inputs at the 3-digit NAICS level from \href{https://www150.statcan.gc.ca/t1/tbl1/en/tv.action?pid=3610021701}{Table 36-10-0217-01} of Statistics Canada.
    \item Industry-level capital expenditures ($\{r_{it} K_{it}\}_{i=1}^N$): We use data from 1961 to 2019 on capital cost at the 3-digit NAICS level from \href{https://www150.statcan.gc.ca/t1/tbl1/en/tv.action?pid=3610021701}{Table 36-10-0217-01} of Statistics Canada.
    \item Industry-level labor expenditures ($\{w_{it} L_{it}\}_{i=1}^N$): We use data from 1961 to 2019 on labor compensation at the 3-digit NAICS level from \href{https://www150.statcan.gc.ca/t1/tbl1/en/tv.action?pid=3610021701}{Table 36-10-0217-01} of Statistics Canada.
    \item Industry-level intermediate input expendiures ($\{\{P_{jt} X_{ijt}\}_{j=1}^N\}_{i=1}^N$): We use data from 1961 to 2019 on the symmetric input-output tables at the 3-digit NAICS level from \href{https://www150.statcan.gc.ca/t1/tbl1/en/tv.action?pid=3610000101}{Table 36-10-0001-01} of Statistics Canada and its previous versions.
\end{enumerate}
\end{comment}

\begin{figure}[h]
    \centering
    \caption{Prices, wages, and TFP growth over sub-periods}
    \begin{subfigure}[b]{0.48\textwidth}
        \centering
        \caption{Prices}
        \includegraphics[width=\textwidth]{price_tfp_growth_period.png}
        \label{fig:price_tfp_growth_period}
    \end{subfigure}
    \hfill
    \begin{subfigure}[b]{0.48\textwidth}
        \centering
        \caption{Wages}
        \includegraphics[width=\textwidth]{wage_tfp_growth_period.png}
        \label{fig:wage_tfp_growth_period}
    \end{subfigure}
    \label{fig:price_wage_tfp_growth_period}
\end{figure}

\begin{figure}[h]
    \centering
    \caption{GDP and TFP growth over sub-periods}
    \begin{subfigure}[b]{0.48\textwidth}
        \centering
        \caption{Nominal GDP}
        \includegraphics[width=\textwidth]{va_tfp_growth_period.png}
        \label{fig:va_tfp_growth_period}
    \end{subfigure}
    \hfill
    \begin{subfigure}[b]{0.48\textwidth}
        \centering
        \caption{Real GDP}
        \includegraphics[width=\textwidth]{real_va_tfp_growth_period.png}
        \label{fig:real_va_tfp_growth_period}
    \end{subfigure}
    \label{fig:gdp_tfp_growth_period}
\end{figure}

\begin{figure}[h]
    \centering
    \caption{Input cost shares and TFP growth}
    \begin{subfigure}[b]{0.48\textwidth}
        \centering
        \caption{Capital}
        \includegraphics[width=\textwidth]{capital_share_tfp_growth.png}
        \label{fig:capital_share_tfp_growth}
    \end{subfigure}
    \hfill
    \begin{subfigure}[b]{0.48\textwidth}
        \centering
        \caption{Labor}
        \includegraphics[width=\textwidth]{labor_share_tfp_growth.png}
        \label{fig:labor_share_tfp_growth}
    \end{subfigure}
    \label{fig:input_share_tfp_growth}
\end{figure}

\begin{figure}[h]
    \centering
    \caption{TFP growth by industry}
    \includegraphics[width=0.9\textwidth]{tfp_growth_industry.png}
    \label{fig:tfp_growth_industry}
\end{figure}

\begin{figure}[h]
    \centering
    \caption{GDP growth by industry}
    \includegraphics[width=0.9\textwidth]{va_growth_industry.png}
    \label{fig:va_growth_industry}
\end{figure}

\begin{figure}[h]
    \centering
    \caption{TFP growth and the Baumol effect (without O\&G)}
    \includegraphics[width=0.8\textwidth]{baumol_no_oge.png}
    \label{fig:baumol_no_oge}
\end{figure}

\end{document}
