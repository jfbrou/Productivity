\documentclass[12pt]{article}
\usepackage{mathpazo, amsmath, amssymb, amsfonts, amsthm, microtype, tikz, graphicx, booktabs, caption, subcaption, bm, xfrac, appendix, setspace, comment, float, nth, fnpct, makecell, multirow, tabularx, threeparttable, titling, array, lipsum, mathtools, tocloft, tocvsec2}
\usepackage[margin=2.75cm]{geometry}
\renewcommand{\baselinestretch}{1.15}
\renewcommand{\arraystretch}{2}
\setlength\droptitle{-2.5cm}
%\setlength\parindent{0cm}
\setlength{\parskip}{0.15cm}
\definecolor{HEC}{rgb}{0, 0.1569, 0.3333}
\usepackage[colorlinks, allcolors=HEC]{hyperref}
\urlstyle{rm}
\captionsetup[figure]{labelfont={color=HEC}}
\captionsetup[table]{labelfont={color=HEC}}
\usepackage[ruled, vlined]{algorithm2e}
\setlength{\abovecaptionskip}{0.5cm}
\usepackage[longnamesfirst]{natbib}

% Command to reference subfigures
\renewcommand{\thesubfigure}{(\alph{subfigure})}
\captionsetup[sub]{labelformat=simple}

% Define the path for the figures
\graphicspath{{../Figures/}}

% Define a command and the path for the tables
\newcommand*\includetables[1]{\input{../Tables/#1.tex}}

% Do not add sections of the main text to the appendix's table of content
\addtocontents{toc}{\protect\setcounter{tocdepth}{-1}}

% Command to comment within lines
\newcommand{\ignore}[1]{}

% Command to center table columns
\newcolumntype{C}[1]{>{\centering\arraybackslash}p{#1}}
\newlength\mcwidth

% Mathematical commands
\DeclareMathOperator*{\argmax}{arg\,max}
\newcommand{\diag}[1]{\text{diag}\left(#1\right)}
\newcommand{\partialof}[2]{\frac{\partial #1}{\partial #2}}
\newcommand{\totalof}[2]{\frac{\text{d} #1}{\text{d} #2}}
\newcommand{\deltaof}[3]{\delta #1 [#2; #3]}
\newcommand{\innerprod}[3]{\langle #1, #2 \rangle_{#3}}
\newtheorem{theorem}{Theorem}
\newtheorem{corollary}{Corollary}
\newtheorem{definition}{Definition}
\newtheorem{proposition}{Proposition}
\newtheorem{lemma}{Lemma}

% Command to resize equations
\newcommand{\resizeeq}[2]{\resizebox{#2\hsize}{!}{$#1$}}

% Command to make a small horizontal space
\newcommand{\smallquad}{\hspace{0.5em}}

% Title
\title{\textbf{The Sectoral Anatomy of the Canadian Productivity Growth Slowdown}\thanks{I thank Nicolas Vincent and Robert Gagné for helpful comments.}}
\author{\href{https://www.jeanfelixbrouillette.com}{Jean-F\'elix Brouillette}\textsuperscript{1}}
\date{\today}

\begin{document}

\maketitle

\footnotetext[1]{HEC Montr\'eal. E-mail: \href{mailto:jean-felix.brouillette@hec.ca}{jean-felix.brouillette@hec.ca}}

\begin{abstract}
    This paper quantifies the forces behind Canada's long-run productivity growth slowdown using an industry-level accounting framework for the period 1961--2019. I decompose aggregate total-factor productivity (TFP) growth into contributions from: (1) within-industry productivity improvements, (2) structural change (Baumol effects), and (3) factor reallocation across industries. The results show that Canada's post-2000 stagnation is driven almost entirely by a sharp decline in within-industry TFP growth. Structural change increasingly weighs on aggregate TFP as the economy shifts toward slower-growing resource service sectors, though its overall magnitude remains modest. Factor reallocation contributes little throughout the sample.
\end{abstract}

\clearpage

\section{Introduction}
\label{s:introduction}

Canada's productivity growth has slowed substantially over the past several decades, with the slowdown becoming especially pronounced at the turn of the millennium.\footnote{While aggregate TFP grew at nearly 1\% annually between 1961 and 1980, it has since collapsed, averaging -0.1\% per year from 2000 to 2019.} This decline has translated directly into weaker wage growth and a noticeable erosion in the purchasing power of Canadian households. Understanding the origins of this slowdown requires disentangling whether it comes from declining productivity growth within industries or from structural shifts in economic activity across industries.

A common narrative attributes Canada's productivity malaise to \citet{Baumol_1967}'s ``cost disease,'' whereby sectors with limited scope for productivity improvements experience rising relative prices, yet capture an increasing share of aggregate expenditures. The expansion of oil sands extraction and the finance, insurance, and real estate (F.I.R.E.) sector is often interpreted through this lens. This raises a critical question: Is Canada's productivity growth slowdown a mechanical consequence of structural change, or does it reflect a broader collapse of TFP growth across industries?

In this paper, I address this question using a growth-accounting framework that decomposes aggregate total-factor productivity (TFP) growth for the Canadian business sector into three components: within-industry productivity improvements, changes in the economy's industrial composition (Baumol effects), and the reallocation of capital and labor across industries. I implement this decomposition using industry-level data (3-digit NAICS) from Statistics Canada for 1961--2019. The analysis extends prior work by (1) allowing factor reallocation to have first-order effects on aggregate TFP growth \citep{Baqaee_Farhi_2019}, and (2) directly comparing the quantitative contributions of sectoral reallocation and within-industry productivity growth.

The results show that Canada's post-2000 productivity stagnation is driven almost entirely by a substantial decline in within-industry TFP growth. The contribution of this component fell from 0.76\% per year (1980--2000) to just 0.12\% per year (2000--2019). Although Baumol effects increasingly weigh on aggregate productivity as activity shifts toward slow-growing resource and service sectors, their overall contribution remains modest, and factor reallocation plays only a minor role throughout the entire sample. These findings mirror the U.S. evidence documented by \citet{Halperin_Mazlish_2024}.

I also address the role of the resource and F.I.R.E. (finance, insurance, and real estate) sectors, which are frequent culprits in discussions of Canadian economic performance. I find that the oil and gas sector contributes disproportionately to the Baumol effect, accounting for approximately half of the structural drag. However, even after excluding that sector, the central finding holds: the contribution of within-industry TFP growth  collapses from over 0.8\% per year prior to 2000 to just 0.2\% thereafter. We obtain similar results when excluding the mining and F.I.R.E. sectors. This suggests that Canada's productivity malaise is not merely a symptom of its resource dependence or a mechanical shift to services, but a widespread failure to deliver efficiency gains across a broad range of industries.

However, these results do not imply that structural change and factor reallocation are irrelevant. As emphasized by \citet{Loertscher_Pujolas_2024} and confirmed in this paper, the oil and gas extraction sector plays a nontrivial role in accounting for Canada's TFP growth divergence with the United States. Yet, it could play an even larger role in the form of a ``Dutch disease'' in disguise. If the appreciation of the Canadian dollar during the resource boom of the 2000s squeezed the market size of tradeable industries, it could have dampened their incentives to invest in research and development (R\&D). Consequently, the resource sector may be indirectly responsible for a broader collapse in within-industry productivity growth.

The remainder of the paper is organized as follows. The following section reviews the relevant literature. Section \ref{s:theoretical framework} presents the theoretical framework underlying our productivity growth decomposition. Section \ref{s:data} describes the data used in the analysis. Section \ref{s:empirical results} presents the empirical results and Section \ref{s:discussion} discusses them. Section \ref{s:conclusion} concludes.

\subsection*{Literature review}

The analysis in this paper is grounded in the foundational literature on aggregate productivity measurement. \citet{Hulten_1978}, building on the work of \citet{Solow_1957} and \citet{Domar_1961}, showed that in efficient economies, aggregate TFP growth can be expressed as a sales-weighted average of industry-level productivity growth. More recent work relaxed the assumptions underlying Hulten's theorem. Notably, \citet{Baqaee_Farhi_2019} show that factor reallocation can have first-order effects on aggregate TFP in distorted economies. I adopt their framework to decompose Canadian TFP growth into three distinct components: within-industry productivity improvements, Baumol effects, and factor reallocation.

This framework allows us to directly engage with the literature on structural change and ``Baumol's cost disease.'' \citet{Baumol_1967} famously theorized that as technologically progressive sectors drive up economy-wide wages, stagnant sectors---unable to offset rising labor costs with efficiency gains---experience increasing relative prices. If consumer demand for these stagnant services is price inelastic, they capture an ever-growing share of nominal expenditures, mechanically weighing on aggregate productivity growth. 

\citet{Nordhaus_2006} tests these predictions empirically by documenting the characteristic symptoms of Baumol's cost disease in U.S. data. More recently, \citet{Halperin_Mazlish_2024} apply a growth-accounting decomposition closely related to the one used in this paper to U.S. data, quantifying the relative roles of within-industry productivity growth, structural change, and factor reallocation in the U.S. productivity slowdown. \citet{Duernecker_Herrendorf_Valentinyi_2023} develop a macroeconomic model of structural change and argue that Baumol's cost disease in the U.S. will be less severe in the future than it was in the past. 

In the Canadian context, a growing body of work documents Canada's recent productivity slowdown, in particular relative to the United States \citep{Dion_2007, Boothe_Roy_2008, Baldwin_Gu_2009, Sharpe_2010}. Recent contributions bring additional structure to the diagnosis. \citet{Conesa_Pujolas_2019} show that the 2002--2014 stagnation cannot be explained by measurement issues or compositional effects. \citet{Loertscher_Pujolas_2024} show that the oil and gas extraction sector explains much of Canada's relative productivity growth gap vis-à-vis the United States. Complementing this evidence, \citet{Alexopoulos_Cohen_2018} and \citet{Mollins_St-Amant_2019} document a slowdown in technology commercialization and ICT adoption, suggesting these phenomena have contributed directly to weaker TFP growth. Finally, \citet{Kiarsi_2025} decomposes Canada's real GDP per capita growth and confirms that the deceleration in living standards is mainly driven by the TFP growth slowdown.

Rather than advancing a single explanation, the contribution of this paper is to quantify the relative importance of competing accounts of Canada's productivity slowdown within a common decomposition framework. In doing so, it extends the existing evidence in three ways: first, by examining a long time horizon (1961--2019); second, by incorporating allocative inefficiencies using the framework of \citet{Baqaee_Farhi_2019}; and third, by directly comparing the contribution of sectoral reallocation with the collapse in within-industry productivity growth.

\section{Theoretical framework}
\label{s:theoretical framework}

To decompose the sources of Canadian productivity growth, I adopt the framework of \citet{Baqaee_Farhi_2019}, which generalizes \citet{Hulten_1978}'s theorem for inefficient economies. This section outlines the key equations underlying the decomposition of aggregate TFP growth.

\paragraph{Input-output definitions.} Consider an economy with $N$ industries indexed by $i \in \{1, \ldots, N\}$ and two factors of production: capital and labor. Let $\mathbf{b}_t$ be the $(N + 2) \times 1$ vector whose $i$-th element is equal to the share of industry $i$ in aggregate nominal value-added:
\begin{equation*}
    b_{it} \equiv \frac{P_{it} Y_{it}}{\sum_{j=1}^N P_{jt} Y_{jt}}, \quad \forall i \in \{1, \ldots, N\}.
\end{equation*}
The first $N$ elements of $\mathbf{b}_t$ correspond to industries, while the last two correspond to capital and labor. Since factors do not enter in final demand, the last two elements of $\mathbf{b}_t$ are equal to zero. Let us further define the cost-based input-output matrix $\bm{\Omega}_t$ of dimension $(N + 2) \times (N + 2)$ whose $(i, j)$-th element is equal to industry $i$'s expenditures on inputs from $j$ as a share of its total expenditures:
\begin{equation*}
    \Omega_{ijt} \equiv \frac{P_{jt} X_{ijt}}{\sum_{k=1}^N P_{kt} X_{ikt}}.
\end{equation*}
The first $N$ rows of $\bm{\Omega}_t$ correspond to industries, while the last two correspond to capital and labor. Since capital and labor require no intermediate inputs, the last two rows of $\bm{\Omega}_t$ are filled with zeros. The Leontief inverse of the cost-based input-output matrix is defined as:
\begin{equation*}
    \bm{\Psi}_t \equiv (\mathbf{I} - \bm{\Omega}_t)^{-1}.
\end{equation*}
Finally, the cost-based Domar weights are defined as:
\begin{equation*}
    \bm{\lambda}_t^{\prime} \equiv \mathbf{b}_t^{\prime} \bm{\Psi}_t.
\end{equation*}
For expositional convenience, we denote by $\lambda_t^K$ and $\lambda_t^L$ the last two elements of $\bm{\lambda}_t^{\prime}$, which measure their importance in final demand, indirectly through the production network of the economy.

\paragraph{Total-factor productivity.} As shown by \citet{Baqaee_Farhi_2019}, aggregate TFP growth in inefficient economies can be calculated as:
\begin{equation*}
    \mathrm{d}\ln(A_t) = \mathrm{d}\ln(Y_t) - \lambda_t^K \mathrm{d}\ln(K_t) - \lambda_t^L \mathrm{d}\ln(L_t)
\end{equation*}
where $A_t$ is aggregate TFP, $Y_t$ is aggregate real value-added, $K_t$ is aggregate capital, and $L_t$ is aggregate labor. The growth rate of aggregate real value-added is given by:
\begin{equation*}
    \mathrm{d}\ln(Y_t) \equiv \sum_{i=1}^N b_{it} \mathrm{d}\ln(Y_{it}).
\end{equation*}
Here, the growth rate of real value-added in industry $i$ is given by:
\begin{equation*}
    \mathrm{d}\ln(Y_{it}) = \mathrm{d}\ln(A_{it}) + \alpha_{it}^K \mathrm{d}\ln(K_{it}) + \alpha_{it}^L \mathrm{d}\ln(L_{it})
\end{equation*}
where $A_{it}$ is industry $i$'s TFP, $K_{it}$ its capital input, $L_{it}$ its labor input, and $\alpha_{it}^K$ and $\alpha_{it}^L$ are the elasticities of its output with respect to capital and labor, respectively. Here, we make the standard assumption of constant returns to scale at the industry level such that $\alpha_{it}^K + \alpha_{it}^L = 1$. Under the additional assumption that the representative firm in industry $i$ minimizes its costs and is a price-taker in input markets, the elasticity of output with respect to the capital input is given by:
\begin{equation*}
    \alpha_{it}^K = \frac{r_{it} K_{it}}{r_{it} K_{it} + w_{it} L_{it}}.
\end{equation*}
The growth rate of aggregate capital is given by:
\begin{equation*}
    \mathrm{d}\ln(K_t) \equiv \sum_{i=1}^N \omega_{it}^K \mathrm{d}\ln(K_{it}) \quad \text{where} \quad \omega_{it}^K \equiv \frac{r_{it} K_{it}}{\sum_{j=1}^N r_{jt} K_{jt}}
\end{equation*}
and the growth rate of aggregate labor is given by:
\begin{equation*}
    \mathrm{d}\ln(L_t) \equiv \sum_{i=1}^N \omega_{it}^L \mathrm{d}\ln(L_{it}) \quad \text{where} \quad \omega_{it}^L \equiv \frac{w_{it} L_{it}}{\sum_{j=1}^N w_{jt} L_{jt}}
\end{equation*}
and where $r_{it}$ and $w_{it}$ are the rental rate of capital and the wage rate in industry $i$ at time $t$, respectively.

\paragraph{Productivity growth accounting.} Combining the above equations, we can express aggregate TFP growth as:
\begin{align*}
    \mathrm{d}\ln(A_t) &= \sum_{i=1}^N b_{it} \mathrm{d}\ln(A_{it}) \\
    &+ \sum_{i=1}^N (b_{it} \alpha_{it}^K - \omega_{it}^K \lambda_t^K) \mathrm{d}\ln(K_{it}) \\
    &+ \sum_{i=1}^N (b_{it} \alpha_{it}^L - \omega_{it}^L \lambda_t^L) \mathrm{d}\ln(L_{it}).
\end{align*}
Adding and subtracting the term $\sum_{i=1}^N b_{it_0} \mathrm{d}\ln(A_{it})$ and summing from time $t_0$ to $t_1$, we can further decompose the cumulative aggregate TFP growth between these two dates into the following components:
\begin{alignat}{2}
    \sum_{t=t_0}^{t_1} \mathrm{d}\ln(A_t) &= \sum_{t=t_0}^{t_1} \sum_{i=1}^N b_{it_0} \mathrm{d}\ln(A_{it}) &&\quad \text{\textcolor{HEC}{Within-industry TFP growth}} \label{eq:productivity} \\
    &+ \sum_{t=t_0}^{t_1} \sum_{i=1}^N (b_{it} - b_{it_0}) \mathrm{d}\ln(A_{it}) &&\quad \text{\textcolor{HEC}{Baumol effects}} \label{eq:baumol} \\
    &+ \sum_{t=t_0}^{t_1} \sum_{i=1}^N (b_{it} \alpha_{it}^K - \omega_{it}^K \lambda_t^K) \mathrm{d}\ln(K_{it}) &&\quad \text{\textcolor{HEC}{Capital reallocation}} \label{eq:capital} \\
    &+ \sum_{t=t_0}^{t_1} \sum_{i=1}^N (b_{it} \alpha_{it}^L - \omega_{it}^L \lambda_t^L) \mathrm{d}\ln(L_{it}) &&\quad \text{\textcolor{HEC}{Labor reallocation}} \label{eq:labor}
\end{alignat}
The interpretation of these four terms is as follows.

\paragraph{1. Within-industry TFP growth.} Term \eqref{eq:productivity} weighs industry-level TFP growth by each industry's initial share of value-added. That is, if we freeze each industry's share of nominal value-added, we capture the contribution of within-industry productivity improvements alone---how much aggregate TFP has grown purely because each industry became more productive over time.

\paragraph{2. Baumol effects.} Term \eqref{eq:baumol} instead weighs industry-level TFP growth by the change in each industry's share of value-added between dates $t_0$ and $t_1$. Therefore, it measures how the changing industrial composition of the economy contributes to aggregate TFP growth. This term is negative when industries with low productivity growth become more important in the economy, or vice-versa.

\paragraph{3. Factor reallocation.} Terms \eqref{eq:capital} and \eqref{eq:labor} capture the contribution of reallocating factors across sectors. It weighs changes in industry-level capital and labor inputs by the difference between two terms. The first term is the product of the industry's share of total value-added and the elasticity of its output with respect to inputs. The second term is the product of the industry's share of total factor expenditures and the elasticity of aggregate output with respect to inputs. Intuitively, moving inputs into industries where they are more productive than the economy-wide average ($\alpha_{it}^K > \lambda_t^K$ or $\alpha_{it}^L > \lambda_t^L$) and where the receiving industries are more important in output than costs ($b_{it} > \omega_{it}^K$ or $b_{it} > \omega_{it}^L$) (i.e., high markup industries) raises aggregate productivity growth.

\section{Data}
\label{s:data}

The empirical analysis covers the Canadian business-sector economy over the period 1961–2019. To implement the decomposition outlined in Section \ref{s:theoretical framework}, I construct a dataset that combines industry-level productivity measures with detailed input–output linkages at the 3-digit North American Industry Classification System (NAICS) level. The data are drawn from two primary sources provided by Statistics Canada.

\paragraph{Productivity and factor inputs.}
Industry-level production and input data are obtained from Statistics Canada Table 36-10-0217-01. This table provides most of the key variables required in the aggregate TFP growth decomposition:
\begin{itemize}
    \item \textbf{Total factor productivity.} Industry-level TFP ($A_{it}$) is measured using Statistics Canada's multifactor productivity (MFP) indices.
    \item \textbf{Factor inputs.} Industry-level factor inputs ($K_{it}$ and $L_{it}$) are measured using Statistics Canada's capital and labor input indices.
    \item \textbf{Value-added weights.} To construct the weights $b_{it}$, I use data on industry-level gross domestic product at basic prices ($P_{it} Y_{it}$).
    \item \textbf{Factor weights.} To construct the weights $\omega_{it}^K$, $\omega_{it}^L$, $\alpha_{it}^K$, and $\alpha_{it}^L$, I use data on industry-level capital and labor costs ($r_{it} K_{it}$ and $w_{it} L_{it}$).
\end{itemize}

\paragraph{Input-output linkages.} To construct the cost-based Domar weights ($\lambda_t^K$ and $\lambda_t^L$), I use the Symmetric Input-Output Tables from Statistics Canada Table 36-10-0001-01 and its historical predecessors. These tables provide the intermediate input expenditures ($\{P_{jt} X_{ijt}\}_{j=1}^N$) between industries. I use these flows to construct the cost-based input-output matrix $\bm{\Omega}_t$ and the resulting Leontief inverse $\bm{\Psi}_t$ for each year in the sample. 

By harmonizing these two data sources at the 3-digit NAICS level, we obtain a balanced panel of industries spanning nearly six decades, allowing us to track the evolution of aggregate TFP through the lens of the \citet{Baqaee_Farhi_2019} framework.

\section{Empirical results}
\label{s:empirical results}

The findings are presented in three steps. First, I examine raw correlations in the data to test for ``symptoms'' of Baumol's cost disease in Canada. Second, I present the main decomposition of aggregate TFP growth to isolate the relative contributions of within-industry TFP growth, structural change, and factor reallocation. Third, I analyze the contributions of different industries, with a specific focus on the outsized role of the resource and F.I.R.E. sectors.

\subsection{The symptoms of a cost disease}

I first assess whether the Canadian economy displays the classic symptoms of unbalanced growth identified by \citet{Baumol_1967}. Baumol's hypothesis rests on a particular transmission mechanism: in industries with rapid productivity growth, efficiency improvements push relative prices downward and wages upward. Stagnant industries, drawing on the same labor pool, must match these wage increases despite lacking corresponding productivity gains. To preserve profit margins under rising labor costs, firms in these sectors are compelled to raise prices. If consumer demand for their services is price inelastic, households continue to purchase them even as costs climb. As a result, these stagnant, high-cost sectors absorb an increasingly large share of nominal GDP.

\begin{figure}[h!]
    \centering
    \caption{Prices, wages, and TFP growth}
    \begin{subfigure}[b]{0.48\textwidth}
        \centering
        \caption{Prices}
        \includegraphics[width=\textwidth]{price_tfp_growth.png}
        \label{fig:price_tfp_growth}
    \end{subfigure}
    \hfill
    \begin{subfigure}[b]{0.48\textwidth}
        \centering
        \caption{Wages}
        \includegraphics[width=\textwidth]{wage_tfp_growth.png}
        \label{fig:wage_tfp_growth}
    \end{subfigure}
    \label{fig:price_wage_tfp_growth}
\end{figure}

Figure \ref{fig:price_wage_tfp_growth} offers strong evidence for the first half of this mechanism. Indeed, Panel \ref{fig:price_tfp_growth} documents a pronounced negative relationship between relative price growth and TFP growth across Canadian industries: sectors with rapid productivity gains (e.g., ``computer and electronic product manufacturing'' or ``wood product manufacturing'') have experienced substantial price declines, while more stagnant sectors (e.g., ``arts, entertainment, and recreation'', ``F.I.R.E'', or ``oil and gas extraction'') have undergone marked price inflation. The presence of the arts, entertainment, and recreation sector is particularly noteworthy, as its labor-intensive production processes famously provided the original motivation for Baumol's cost disease. By contrast, Panel \ref{fig:wage_tfp_growth} shows no systematic association between wage growth and productivity growth, in line with Baumol's hypothesis.

Figure \ref{fig:gdp_tfp_growth} corroborates the second part of the mechanism: that ``stagnant'' sectors are indeed capturing a larger share of the market. Panel \ref{fig:va_tfp_growth} shows a pronounced negative relationship between TFP growth and nominal GDP growth. As technologically progressive manufacturing industries contract in relative value because of declining prices, spending shifts toward low-productivity resource and service sectors. Panel \ref{fig:real_va_tfp_growth} reveals a far weaker correlation between real GDP growth and TFP growth, indicating that demand for stagnant sectors is indeed relatively inelastic. Taken together, these patterns suggest that the Canadian economy exhibits the key symptoms of Baumol's cost disease.

\begin{figure}[h!]
    \centering
    \caption{GDP and TFP growth}
    \begin{subfigure}[b]{0.48\textwidth}
        \centering
        \caption{Nominal GDP}
        \includegraphics[width=\textwidth]{va_tfp_growth.png}
        \label{fig:va_tfp_growth}
    \end{subfigure}
    \hfill
    \begin{subfigure}[b]{0.48\textwidth}
        \centering
        \caption{Real GDP}
        \includegraphics[width=\textwidth]{real_va_tfp_growth.png}
        \label{fig:real_va_tfp_growth}
    \end{subfigure}
    \label{fig:gdp_tfp_growth}
\end{figure}

In Appendix \ref{a:appendix}, I show that these patterns are remarkably stable over time. Replicating Figures \ref{fig:price_wage_tfp_growth} and \ref{fig:gdp_tfp_growth} for the periods 1961--1980, 1980--2000, and 2000--2019 reveals that the same relationships persist across all three intervals. This robustness suggests that the forces underlying Baumol's cost disease are not episodic but structural: for decades, sectors with slow productivity growth have consistently seen their prices rise and their market shares expand. I also show that input (labor and capital) cost shares are not systematically related to TFP growth.

\subsection{The productivity growth diagnosis}

To quantify these forces, Table \ref{tab:tfp_decomposition} decomposes aggregate TFP growth into its three constitutive parts: within-industry productivity growth, structural change (\textit{Baumol effects}), and factor (capital and labor) reallocation. The table reports this decomposition for the full sample (1961--2019) as well as for three sub-intervals: 1961--1980, 1980--2000, and 2000--2019.\footnote{In each subperiod, value-added shares are fixed at the initial year of the interval. Consequently, the Baumol term reflects incremental structural change within the period and is mechanically smaller than the cumulative Baumol effect measured over the full sample.}

\includetables{tfp_decomposition}

Across the entire sample, aggregate TFP growth averages 0.5\% per year and is overwhelmingly propelled by within-industry productivity gains (0.78 percentage points). The Baumol component contributes –0.27 percentage points, indicating that the economy's gradual shift toward slow-productivity-growth sectors exerts a sizable drag on aggregate TFP growth. Factor reallocation plays only a minor role: capital and labor reallocations contribute –0.11 and 0.10 percentage points, respectively, largely offsetting one another. Figure \ref{fig:tfp_decomposition} visually summarizes these results.

\begin{figure}[h!]
    \centering
    \caption{TFP growth decomposition}
    \includegraphics[width=0.8\textwidth]{tfp_decomposition.png}
    \label{fig:tfp_decomposition}
\end{figure}

However, the full-sample decomposition masks a striking shift over time. In the early period (1961--1980), aggregate TFP growth averages nearly 1\% per year, but by 2000--2019 it falls to –0.1\% per year. The decomposition highlights three central insights behind this TFP growth collapse.

\paragraph{The collapse occurs within industries.} The first row of Table \ref{tab:tfp_decomposition} shows that the slowdown is clearly a within-industry phenomenon. From 1961 to 1980, this term contributed 1.11\% to annual TFP growth; by 1980--2000, its contribution had slowed to 0.76\%, and after 2000 it fell sharply to just 0.12\%, explaining most of the decline. These patterns indicate that the core issue is not a structural reallocation toward low-TFP-growth sectors but a widespread loss of momentum in productivity improvements within industries.

\begin{figure}[h!]
    \centering
    \caption{TFP growth and the Baumol effect}
    \includegraphics[width=0.8\textwidth]{baumol.png}
    \label{fig:baumol}
\end{figure}

\paragraph{The Baumol effect is a steady headwind.} The Baumol term is persistently negative, confirming that economic activity continues to shift toward slower-growing sectors. Its magnitude, however, evolves only modestly---from –0.08\% in 1961--1980 to –0.16\% in 1980--2000 and –0.20\% in 2000--2019. Although this structural drag aggravates the slowdown, it is not its principal driver. Figure \ref{fig:baumol} illustrates the cumulative impact of this drag: absent this adverse reallocation, aggregate TFP would be roughly 12\% higher by 2019.

\paragraph{Factor reallocation plays a minor role.} The contributions from capital and labor reallocation are small and largely offset each other, indicating that the slowdown cannot be blamed on barriers that would impede resources from flowing toward high-productivity sectors. This is noteworthy in light of growing concerns that market power has been increasing in Canada, potentially distorting competitive pressures and resource allocation \citep{CBC_2023}. The decomposition suggests that, despite these concerns, factor misallocation is not the primary force behind the aggregate TFP growth decline.

\subsection{The industrial anatomy of the slowdown}

Although the aggregate decomposition indicates that the Baumol effect is not the primary driver of Canada's productivity slowdown, examining the individual contributions of different industries remains informative. The framework developed in Section \ref{s:theoretical framework} allows aggregate TFP growth to be decomposed additively into industry-specific contributions. Figure \ref{fig:tfp_contribution_industry} presents this decomposition over the full sample period. The results reveal pronounced heterogeneity across industries: retail trade and several manufacturing industries make positive contributions to aggregate TFP growth, whereas oil and gas extraction, F.I.R.E., and mining display substantial negative contributions.

\begin{figure}[h!]
    \centering
    \caption{Industrial contributions to TFP growth (1961--2019)}
    \includegraphics[width=0.9\textwidth]{tfp_contribution_industry.png}
    \label{fig:tfp_contribution_industry}
\end{figure}

The sources of these negative contributions become clearer when comparing industry-level TFP growth (Figure \ref{fig:tfp_growth_industry}) with nominal GDP growth (Figure \ref{fig:va_growth_industry}) in Appendix \ref{a:appendix}. The sectors with the largest negative contributions---especially oil and gas extraction---are those experiencing the deepest TFP contractions over the entire sample. Yet these same industries have expanded markedly in nominal terms. This decoupling, in which a sector becomes less efficient while claiming a larger share of the economy, is the defining signature of Baumol's cost disease.

While Figure \ref{fig:tfp_contribution_industry} highlights which industries ultimately pulled aggregate TFP growth up or down over the entire sample (1961--2019), it does not directly indicate where the slowdown itself originated. To isolate the sources of the deceleration, Figure \ref{fig:tfp_growth_change_industry} instead focuses on changes in industry-level TFP growth rates over time, plotting for each industry the difference in average annual TFP growth between 1961--1980 and 2000--2019.

\begin{figure}[h!]
    \centering
    \caption{Industry-level TFP growth changes (1961--1980 vs. 2000--2019)}
    \includegraphics[width=0.9\textwidth]{tfp_growth_change_industry.png}
    \label{fig:tfp_growth_change_industry}
\end{figure}

The resulting pattern reveals that the slowdown is broad-based rather than driven by a few industries. The distribution is sharply skewed to the left: only a handful of industries experience faster productivity growth in the post-2000 period, while the vast majority exhibit a deceleration, often substantial. This pattern is particularly pronounced in manufacturing, where many industries shift from positive TFP growth before 1980 to near-zero or negative growth after 2000.

\paragraph{Resource and F.I.R.E. sectors.} Given the outsized contribution of the resource sector shown in Figure \ref{fig:tfp_contribution_industry}, a natural question is whether Canada's productivity slowdown is primarily a resource-sector phenomenon. To investigate this, Table \ref{tab:tfp_decomposition_no_oge} replicates the aggregate decomposition after excluding oil and gas extraction. The results indicate that the resource sector does amplify the slowdown: removing it raises post-2000 aggregate TFP growth from –0.09\% to a modestly positive 0.11\%. Nevertheless, the broader pattern remains largely intact. Even without oil and gas, within-industry productivity growth declines sharply, falling from 0.82\% to just 0.21\%.

\includetables{tfp_decomposition_no_oge}

Table \ref{tab:tfp_decomposition_no_stagnant} in Appendix \ref{a:appendix} extends this exercise by further excluding the mining and F.I.R.E. sectors---two industries that, like oil and gas, have displayed a combination of weak productivity growth but strong nominal GDP growth. In particular, the F.I.R.E. sector's growing economic importance is closely tied to Canada's elevated housing prices, which have become a central part of the country's broader productivity debate. Yet even after removing these additional stagnant sectors, the collapse in within-industry productivity growth remains strikingly persistent. This finding underscores that while sector-specific headwinds are part of the story, they sit atop a deeper, economy-wide slowdown in within-industry efficiency growth.

\subsection{A comparison with the United States}
\label{s:usa}

To place the Canadian experience in perspective, I compare my results with those for the United States. \citet{Halperin_Mazlish_2024} apply the same decomposition to U.S. data over the 1947--2016 period to analyze the post-1973 productivity slowdown, which is presented in Table \ref{tab:tfp_decomposition_usa}. The parallels between the two economies are striking.

First, the dominant source of stagnation is the same on both sides of the border. \citet{Halperin_Mazlish_2024} show that the U.S. productivity slowdown is driven overwhelmingly by a collapse in the within-industry component; they estimate that this contribution fell by roughly 0.75 percentage points after 1973. This mirrors my results for Canada and suggests that the slowdown is not an idiosyncratic national phenomenon but part of a broader global deceleration in technological progress---a pattern consistent with the view that ``ideas are getting harder to find'' \citep{Bloom_Jones_VanReenen_Webb_2020}.

\begin{table}[h!]
    \centering
    \begin{threeparttable}
        \caption{U.S. TFP growth decomposition}
        \begin{tabular}{lcccc}
        \hline
        \hline
        & & 1947--2016 & Pre-1973 & Post-1973 \\
        \hline
        Within-industry TFP growth & & 1.04\% & 1.52\% & 0.77\% \\
        Baumol effects & & -0.28\% & -0.13\% & -0.37\% \\
        Capital reallocation & & -0.01\% & -0.04\% & -0.01\% \\
        Labor reallocation & & -0.00\% & -0.00\% & -0.01\% \\
        \hline
        Total & & 0.75\% & 1.35\% & 0.41\% \\
        \hline
        \hline
        \end{tabular}
        \begin{tablenotes}[flushleft]
        \footnotesize
        \item \textit{Note:} This table reproduces the U.S. TFP growth decomposition from \citet{Halperin_Mazlish_2024} using their reported estimates. The sample covers the period 1947--2016, with a break at 1973 to capture the post-oil shock productivity slowdown.
        \end{tablenotes}
        \label{tab:tfp_decomposition_usa}
    \end{threeparttable}
\end{table}

Second, both countries experience a persistent yet secondary drag from structural change. \citet{Halperin_Mazlish_2024} estimate that Baumol's cost disease accounts for around one-quarter of the U.S. slowdown, subtracting about 25 basis points annually. The Canadian decomposition yields a remarkably similar pattern: the Baumol term reduces growth by about 0.27 percentage points per year over the full sample. In each case, the shift toward stagnant service sectors acts as a steady headwind, not the principal driver of the post-2000 decline. Finally, the evidence on factor reallocation is broadly comparable, suggesting that static misallocation is unlikely to be the primary force behind North America's productivity slump.

It is worth noting that the collapse of the within-industry component in the U.S. has proven robust to scrutiny regarding statistical methodology. A prominent counter-hypothesis posits that official price indices fail to fully capture quality improvements, leading statisticians to misinterpret value derived from higher-quality products as pure price inflation \citep{Bils_Klenow_2001}. If the price deflators used to adjust nominal expenditures are overstated because they miss these quality improvements, real output---and consequently productivity---will be systematically understated. 

However, \citet{Byrne_Fernald_Reinsdorf_2017} emphasize that for mismeasurement to account for the productivity slowdown, the extent of mismeasurement would need to have intensified over time. Their evidence points in the opposite direction: the understatement of IT's contribution does not appear to have worsened since the early 2000s and was, if anything, more pronounced in earlier decades. They also note that while the proliferation of ``free'' digital services has undeniably expanded consumer welfare, these benefits primarily accrue to non-market leisure time and are not large enough to explain the slowdown in business sector TFP growth. Taken together, these findings suggest that the North American productivity deceleration reflects a real phenomenon rather than a mere statistical artifact.

\section{Discussion}
\label{s:discussion}

The main accounting exercise of this paper yields a clear conclusion: the aggregate productivity growth slowdown in Canada is driven by a decline in the pace of within-industry efficiency improvements. The direct contributions of structural change (Baumol effects) and factor reallocation appear modest by comparison. However, interpreting these components as entirely independent of each other may be misleading.

A salient example is the role of the resource sector. Although the decomposition in Section \ref{s:empirical results} isolates the direct ``Baumol effect'' of this sector, it abstracts from the general equilibrium consequences of a resource boom on other industries. If the Canadian economy suffers from a form of the ``Dutch disease''---where strong resource exports drive up the real exchange rate---technologically progressive tradeable sectors (e.g., manufacturing) may become less competitive in international markets. This contraction in their effective market size may thus reduce the return on investments in research and development (R\&D), consistent with the findings of \citet{Aghion_Bergeaud_Lequien_Melitz_2024}. Under this interpretation, a resource boom would show up in the above accounting framework not just as a Baumol effect, but as a collapse in the within-industry TFP growth contribution of exporting industries.

There is substantial evidence supporting this channel in the Canadian context. \citet{Baldwin_Yan_2015} and \citet{DaSilva_Vincent_2012} show that access to larger export markets raises productivity by allowing Canadian firms to exploit economies of scale and product specialization, intensifying competitive pressures that reward efficiency, and strengthening incentives to invest and innovate. Consistent with this mechanism, \citet{Baldwin_Gu_Yan_2013} attribute Canada's manufacturing productivity slowdown to weakening export growth and rising excess capacity---patterns aligned with exchange-rate pressures during the commodity boom of the 2000s. Figure \ref{fig:tfp_growth_change_industry} accords with this interpretation: manufacturing industries experience some of the largest declines in TFP growth when comparing the pre-1980 and post-2000 periods.

A similar logic applies to the role of market power. The reallocation terms in the decomposition only capture the static cost of resource misallocation---the efficiency lost because capital and labor are not directed towards the industries where their marginal product is highest. They do not, however, capture the dynamic consequences of market power. R\&D typically entails high fixed costs; therefore, the incentive to invest in technology adoption or productivity improvements is increasing in the scale of production. If rising market power leads firms to restrict output to support prices, it simultaneously reduces the scale over which they can amortize the costs of innovation \citep{Arrow_1962}. Consequently, distortions that appear quantitatively small in the static reallocation terms may nonetheless be responsible for a significant portion of the decline in the within-industry residual by depressing firm-level incentives to develop or adopt new technologies.

This perspective is consistent with a broader Canadian literature documenting a gradual erosion of competitive intensity. Recent evidence from \citet{CBC_2023} points to rising concentration, weaker business dynamism, and higher markups across a range of industries since 2000---patterns indicative of weaker competitive intensity. These developments are unlikely to register strongly in the static reallocation terms of the decomposition, yet they may still matter greatly for productivity growth by slowing the development, diffusion, and adoption of new technologies.

Therefore, while my results highlight the \textit{proximate} cause of the slowdown as a failure of within-industry efficiency, the \textit{ultimate} causes may still be rooted in the structural composition of the economy and the competitive environment. Policy interventions addressing market power, misallocation, or sectoral imbalances may thus yield productivity gains that exceed the modest direct contributions inferred in the static decomposition.

\section{Conclusion}
\label{s:conclusion}

This paper revisits the puzzle of Canada's productivity growth slowdown through the lens of a general growth-accounting framework. The analysis studies whether Canada's recent productivity stagnation reflects an inevitable shift toward services à la Baumol, a deterioration in the allocation of resources across industries, or a broad-based slowdown of TFP growth within industries.

The results attribute the post-2000 slowdown almost entirely to a collapse in within-industry productivity growth. This component's contribution to aggregate TFP growth fell from an annual average of 0.76\% (1980--2000) to 0.12\% (2000--2019). By contrast, while I find evidence that the Canadian economy exhibits symptoms of Baumol's cost disease, its direct contribution to the slowdown is relatively modest. Similarly, the direct contribution of factor reallocation remains quantitatively small throughout the sample.

The analysis also sheds light on the role of the resource sector, frequently cited as a source of Canada's weak productivity performance \citep{Loertscher_Pujolas_2024}. I find that the oil and gas industry accounts for approximately half of the economy-wide Baumol effect. However, even in the non-resource economy, within-industry productivity growth has collapsed. This suggests that the productivity crisis is not a localized symptom of the resource boom, but a broad-based malaise.

These results have important policy implications. If the productivity slowdown primarily reflected worsening allocative efficiency, policy efforts would naturally focus on strengthening competition and reducing barriers to factor reallocation. If it were instead driven by a pure Baumol effect, the deceleration could be interpreted as a largely unavoidable consequence of structural change. The evidence points to neither explanation as dominant. Rather, the overwhelming contribution of within-industry productivity declines suggests a broad weakening of incentives for innovation, technology adoption, and productivity- or quality-enhancing investment.

More broadly, these findings highlight the need for deeper inquiry into the microeconomic origins of the within-industry productivity collapse. Aggregate decompositions are informative about \textit{where} productivity growth is failing, but they cannot reveal the firm-level mechanisms responsible for the slowdown. Addressing this gap requires rich micro-data that track firms with sufficient detail to distinguish among competing explanations. By opening this microeconomic black box, future research can move beyond diagnosis toward a clearer understanding of---and ultimately policy responses to---the structural forces that have undermined Canadian productivity growth.

\clearpage

\bibliography{references}
\bibliographystyle{aer}

\clearpage
\appendix
\section{Appendix}
\label{a:appendix}
\numberwithin{equation}{section}
\renewcommand\thefigure{\thesection.\arabic{figure}}
\renewcommand\thetable{\thesection.\arabic{table}}

\begin{figure}[h]
    \centering
    \caption{Prices, wages, and TFP growth over sub-periods}
    \begin{subfigure}[b]{0.48\textwidth}
        \centering
        \caption{Prices}
        \includegraphics[width=\textwidth]{price_tfp_growth_period.png}
        \label{fig:price_tfp_growth_period}
    \end{subfigure}
    \hfill
    \begin{subfigure}[b]{0.48\textwidth}
        \centering
        \caption{Wages}
        \includegraphics[width=\textwidth]{wage_tfp_growth_period.png}
        \label{fig:wage_tfp_growth_period}
    \end{subfigure}
    \label{fig:price_wage_tfp_growth_period}
\end{figure}

\begin{figure}[h]
    \centering
    \caption{GDP and TFP growth over sub-periods}
    \begin{subfigure}[b]{0.48\textwidth}
        \centering
        \caption{Nominal GDP}
        \includegraphics[width=\textwidth]{va_tfp_growth_period.png}
        \label{fig:va_tfp_growth_period}
    \end{subfigure}
    \hfill
    \begin{subfigure}[b]{0.48\textwidth}
        \centering
        \caption{Real GDP}
        \includegraphics[width=\textwidth]{real_va_tfp_growth_period.png}
        \label{fig:real_va_tfp_growth_period}
    \end{subfigure}
    \label{fig:gdp_tfp_growth_period}
\end{figure}

\begin{figure}[h]
    \centering
    \caption{Input cost shares and TFP growth}
    \begin{subfigure}[b]{0.48\textwidth}
        \centering
        \caption{Capital}
        \includegraphics[width=\textwidth]{capital_share_tfp_growth.png}
        \label{fig:capital_share_tfp_growth}
    \end{subfigure}
    \hfill
    \begin{subfigure}[b]{0.48\textwidth}
        \centering
        \caption{Labor}
        \includegraphics[width=\textwidth]{labor_share_tfp_growth.png}
        \label{fig:labor_share_tfp_growth}
    \end{subfigure}
    \label{fig:input_share_tfp_growth}
\end{figure}

\begin{figure}[h]
    \centering
    \caption{TFP growth by industry}
    \includegraphics[width=0.9\textwidth]{tfp_growth_industry.png}
    \label{fig:tfp_growth_industry}
\end{figure}

\begin{figure}[h]
    \centering
    \caption{GDP growth by industry}
    \includegraphics[width=0.9\textwidth]{va_growth_industry.png}
    \label{fig:va_growth_industry}
\end{figure}

\begin{figure}[h]
    \centering
    \caption{TFP growth and the Baumol effect (without O\&G)}
    \includegraphics[width=0.8\textwidth]{baumol_no_oge.png}
    \label{fig:baumol_no_oge}
\end{figure}

\includetables{tfp_decomposition_no_stagnant}

\end{document}
