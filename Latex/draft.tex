\documentclass[12pt]{article}
\usepackage{mathpazo, amsmath, amssymb, amsfonts, amsthm, microtype, tikz, titling, graphicx, booktabs, caption, subcaption, bm, xfrac, appendix, setspace, comment, float, nth, fnpct, makecell, multirow, tabularx, threeparttable, titling, array, lipsum, mathtools, tocloft, tocvsec2}
\usepackage[margin=2.75cm]{geometry}
\renewcommand{\baselinestretch}{1.15}
\renewcommand{\arraystretch}{2}
\setlength\droptitle{-2.5cm}
%\setlength\parindent{0cm}
\setlength{\parskip}{0.15cm}
\definecolor{HEC}{rgb}{0, 0.1569, 0.3333}
\usepackage[colorlinks, allcolors=HEC]{hyperref}
\urlstyle{rm}
\captionsetup[figure]{labelfont={color=HEC}}
\captionsetup[table]{labelfont={color=HEC}}
\usepackage[ruled, vlined]{algorithm2e}
\setlength{\abovecaptionskip}{0.5cm}
\usepackage[longnamesfirst]{natbib}

% Command to reference subfigures
\renewcommand{\thesubfigure}{(\alph{subfigure})}
\captionsetup[sub]{labelformat=simple}

% Define the path for the figures
\graphicspath{{../Figures/}}

% Define a command and the path for the tables
\newcommand*\includetables[1]{\input{../Tables/#1.tex}}

% Do not add sections of the main text to the appendix's table of content
\addtocontents{toc}{\protect\setcounter{tocdepth}{-1}}

% Command to comment within lines
\newcommand{\ignore}[1]{}

% Command to center table columns
\newcolumntype{C}[1]{>{\centering\arraybackslash}p{#1}}
\newlength\mcwidth

% Mathematical commands
\DeclareMathOperator*{\argmax}{arg\,max}
\newcommand{\diag}[1]{\text{diag}\left(#1\right)}
\newcommand{\partialof}[2]{\frac{\partial #1}{\partial #2}}
\newcommand{\totalof}[2]{\frac{\text{d} #1}{\text{d} #2}}
\newcommand{\deltaof}[3]{\delta #1 [#2; #3]}
\newcommand{\innerprod}[3]{\langle #1, #2 \rangle_{#3}}
\newtheorem{theorem}{Theorem}
\newtheorem{corollary}{Corollary}
\newtheorem{definition}{Definition}
\newtheorem{proposition}{Proposition}
\newtheorem{lemma}{Lemma}

% Command to resize equations
\newcommand{\resizeeq}[2]{\resizebox{#2\hsize}{!}{$#1$}}

% Command to make a small horizontal space
\newcommand{\smallquad}{\hspace{0.5em}}

% Title
\title{\textbf{Where Did Canada's Productivity Growth Go?}\thanks{We are grateful to Edward Xu for excellent research assistance.}}
\author{
\href{https://www.jeanfelixbrouillette.com}{Jean-F\'elix Brouillette}\textsuperscript{1} \and
\href{https://sites.google.com/view/nicolasvincent/home}{Nicolas Vincent}\textsuperscript{2}
}
\date{\today}

\begin{document}

\maketitle

\footnotetext[1]{HEC Montr\'eal. E-mail: \href{mailto:jean-felix.brouillette@hec.ca}{jean-felix.brouillette@hec.ca}}
\footnotetext[2]{HEC Montr\'eal. E-mail: \href{mailto:nicolas.vincent@hec.ca}{nicolas.vincent@hec.ca}}

\begin{abstract}
    This paper quantifies the forces behind Canada's long-run productivity growth slowdown using an industry-level accounting framework for the period 1961--2019. We decompose aggregate total-factor productivity (TFP) growth into contributions from: (1) within-industry productivity improvements, (2) structural change (Baumol effects), and (3) factor reallocation across industries. The results show that Canada's post-2000 stagnation is driven almost entirely by a sharp decline in within-industry TFP growth. Structural change increasingly weighs on aggregate TFP as the economy shifts toward slower-growing service sectors, though its overall magnitude remains modest. Factor reallocation contributes little throughout the sample.
\end{abstract}

\clearpage

\section{Introduction}
\label{s:introduction}

Canada's productivity growth has slowed substantially over the past several decades, with the slowdown becoming especially pronounced after 2000. This decline has translated directly into weaker wage growth and a noticeable erosion in the purchasing power of Canadian households. Understanding the origins of this slowdown requires disentangling whether it comes from declining productivity growth within industries or from structural shifts in economic activity across industries.

In this paper, we address this question using a growth-accounting framework that decomposes aggregate total-factor productivity (TFP) growth into three components: within-industry productivity improvements, changes in the economy's industrial composition (Baumol effects), and the reallocation of capital and labor across industries. Building on the approach of \citet{Halperin_Mazlish_2024}, the decomposition isolates how much TFP growth can be attributed to technological and efficiency gains within industries and how much reflects shifts in value-added or inputs toward industries with faster or slower productivity growth.

Distinguishing among these forces is critical because each carries different implications. If the slowdown reflects structural shifts toward sectors with low productivity growth (i.e., Baumol's cost disease), then weaker TFP growth may be an unavoidable consequence of preferences or technological constraints outside the reach of policy. If it reflects adverse patterns of factor reallocation, such as productive industries drawing too little capital or labor due to rising market power, the slowdown may instead reflect misallocation that policy could address. And if the decline occurs within industries, it points toward underinvestment in technology adoption and innovation, or declining business dynamism. Identifying whether Canada's slowdown is driven by within-industry forces, sectoral shifts, or factor reallocation is thus important for determining whether the TFP growth decline is a structural inevitability or a policy-relevant challenge.

We implement our growth-accounting decomposition using detailed 3-digit NAICS data on value-added, multifactor productivity, and capital and labor inputs from Statistics Canada for 1961–2019. In line with \citet{Baqaee_Farhi_2019}, we do not impose the efficiency conditions required for Hulten's theorem to hold, allowing factor reallocation to have first-order effects on aggregate productivity growth. The results show that Canada's post-2000 productivity stagnation is driven overwhelmingly by a sharp decline in within-industry TFP growth. Although Baumol effects increasingly weigh on aggregate productivity as activity shifts toward slow-growing sectors, their overall contribution remains modest, and factor reallocation plays only a minor role throughout the sample.

The remainder of the paper is organized as follows. The following section reviews the relevant literature. Section \ref{s:theoretical framework} presents the theoretical framework underlying our productivity growth decomposition. Section \ref{s:data} describes the data used in the analysis. Section \ref{s:empirical results} presents the empirical results. Section \ref{s:conclusion} concludes.

\subsection*{Literature review}

Our analysis is grounded in the foundational literature on productivity measurement. \citet{Hulten_1978}, building on the work of \citet{Solow_1957} and \citet{Domar_1961}, showed that in efficient economies, aggregate TFP growth can be expressed as a sales-weighted average of industry-level productivity growth. More recent work relaxed the assumptions underlying Hulten's theorem. Notably, \citet{Baqaee_Farhi_2019} show that factor reallocation can have first-order effects on aggregate TFP in distorted economies. We adopt their framework to decompose Canadian TFP growth into three distinct components: within-industry productivity improvements, Baumol effects, and factor reallocation.

This framework allows us to directly engage with the literature on structural change and ``Baumol's cost disease.'' \citet{Baumol_1967} famously theorized that as technologically progressive sectors drive up economy-wide wages, stagnant sectors--unable to offset rising labor costs with efficiency gains--experience increasing relative prices. If consumer demand for these stagnant services is price inelastic, they capture an ever-growing share of nominal GDP, mechanically dampening aggregate productivity growth. \citet{Nordhaus_2006} and \citet{Halperin_Mazlish_2024} took this insight to the data by measuring sectoral contributions to U.S. productivity growth and highlighting the central role of this sectoral reallocation. \citet{Duernecker_Herrendorf_Valentinyi_2023} develop a macroeconomic model of structural change and argue that Baumol's cost disease in the U.S. will be less severe in the future than it was in the past. 

Within the Canadian context, our work contributes to the analysis of \citet{Conesa_Pujolas_2019}. They document Canada's weak TFP growth from 2002 to 2014, and use a standard growth-accounting framework to rule out measurement and compositional errors. We first contribute to their work by extending the observation window to cover 1961 through 2019, capturing the long-run evolution of Canadian productivity growth. Second, we use a theoretical framework that explicitly allows for allocative inefficiencies. Third, we decompose aggregate TFP growth into within-industry improvements, factor reallocation, and Baumol effects, providing a granular perspective on the forces underlying the aggregate trends they identify.

We also contribute to the debate on the resource sector's role in this stagnation, as highlighted by \citet{Loertscher_Pujolas_2024}. While they find that the oil and gas sector is entirely responsible for Canada's \textit{relative} TFP growth slowdown compared to the U.S. (2001--2018), our analysis focuses on the forces behind Canada's \textit{absolute} TFP growth decline. Allowing for inefficiencies (which may be particularly relevant in resource-intensive industries) to have first-order consequences, we find that while the oil and gas sector indeed accounts for approximately half of the total Baumol effect in Canada, this effect itself is quantitatively too small to explain the broader productivity slowdown. This suggests that the roots of Canada's productivity stagnation extend beyond the structural drag of the energy sector alone.

\section{Theoretical framework}
\label{s:theoretical framework}

To decompose the sources of Canadian productivity growth, we adopt the general equilibrium growth-accounting framework of \citet{Baqaee_Farhi_2019}. This approach allows us to aggregate industry-level shocks into economy-wide TFP growth while explicitly accounting for input-output linkages and allocative inefficiencies.

\paragraph{Input-output definitions.} We consider an economy with $N$ industries indexed by $i \in \{1, \ldots, N\}$ and two factors of production: capital and labor. Let $\mathbf{b}_t$ be the $(N + 2) \times 1$ vector whose $i$-th element is equal to the share of industry $i$ in aggregate nominal value-added:
\begin{equation*}
    b_{it} \equiv \frac{P_{it} Y_{it}}{\sum_{j=1}^N P_{jt} Y_{jt}}, \quad \forall i \{1, \ldots, N\}.
\end{equation*}
The first $N$ elements of $\mathbf{b}_t$ correspond to industries, while the last two correspond to capital and labor. Since factors do not enter in final demand, the last two elements of $\mathbf{b}_t$ are equal to zero. Let us further define the cost-based input-output matrix $\bm{\Omega}_t$ of dimension $(N + 2) \times (N + 2)$ whose $(i, j)$-th element is equal to industry $i$'s expenditures on inputs from $j$ as a share of its total expenditures:
\begin{equation*}
    \Omega_{ijt} \equiv \frac{P_{jt} X_{ijt}}{\sum_{k=1}^N P_{kt} X_{ikt}}.
\end{equation*}
The first $N$ rows of $\bm{\Omega}_t$ correspond to industries, while the last two correspond to capital and labor. Since capital and labor require no intermediate inputs, the last two rows of $\bm{\Omega}_t$ are filled with zeros. The Leontief inverse of the cost-based input-output matrix is defined as:
\begin{equation*}
    \bm{\Psi}_t \equiv (\mathbf{I} - \bm{\Omega}_t)^{-1}.
\end{equation*}
Finally, we define the cost-based Domar weights as:
\begin{equation*}
    \bm{\lambda}_t^{\prime} \equiv \mathbf{b}_t^{\prime} \bm{\Psi}_t.
\end{equation*}
For expositional convenience, we denote by $\lambda_t^K$ and $\lambda_t^L$ the last two elements of $\bm{\lambda}_t^{\prime}$, which measure their importance in final demand, indirectly through the production network of the economy.

\paragraph{Total-factor productivity.} As shown by \citet{Baqaee_Farhi_2019}, aggregate TFP growth in inefficient economies can be calculated as:
\begin{equation*}
    \mathrm{d}\ln(A_t) = \mathrm{d}\ln(Y_t) - \lambda_t^K \mathrm{d}\ln(K_t) - \lambda_t^L \mathrm{d}\ln(L_t)
\end{equation*}
where $A_t$ is aggregate TFP, $Y_t$ is aggregate real value-added, $K_t$ is aggregate capital, and $L_t$ is aggregate labor. The growth rate of aggregate real value-added is given by:
\begin{equation*}
    \mathrm{d}\ln(Y_t) \equiv \sum_{i=1}^N b_{it} \mathrm{d}\ln(Y_{it}).
\end{equation*}
Here, the growth rate of real value-added in industry $i$ is given by:\footnote{Here, we make the assumption of constant returns to scale at the industry level.}
\begin{equation*}
    \mathrm{d}\ln(Y_{it}) = \mathrm{d}\ln(A_{it}) + \alpha_{it}^K \mathrm{d}\ln(K_{it}) + \alpha_{it}^L \mathrm{d}\ln(L_{it})
\end{equation*}
where $A_{it}$ is industry $i$'s TFP, $K_{it}$ its capital input, $L_{it}$ its labor input, and $\alpha_{it}^K$ and $\alpha_{it}^L$ are the elasticities of its output with respect to capital and labor, respectively. Here, we make the assumption of constant returns to scale at the industry level such that $\alpha_{it}^K + \alpha_{it}^L = 1$. Under the additional assumption that the representative firm in industry $i$ minimizes its costs and is a price-taker in input markets, the elasticity of output with respect to the capital input is given by:
\begin{equation*}
    \alpha_{it}^K = \frac{r_{it} K_{it}}{r_{it} K_{it} + w_{it} L_{it}}.
\end{equation*}
The growth rate of aggregate capital is given by:
\begin{equation*}
    \mathrm{d}\ln(K_t) \equiv \sum_{i=1}^N \omega_{it}^K \mathrm{d}\ln(K_{it}) \quad \text{where} \quad \omega_{it}^K \equiv \frac{r_{it} K_{it}}{\sum_{j=1}^N r_{jt} K_{jt}}
\end{equation*}
and the growth rate of aggregate labor is given by:
\begin{equation*}
    \mathrm{d}\ln(L_t) \equiv \sum_{i=1}^N \omega_{it}^L \mathrm{d}\ln(L_{it}) \quad \text{where} \quad \omega_{it}^L \equiv \frac{w_{it} L_{it}}{\sum_{j=1}^N w_{jt} L_{jt}}
\end{equation*}
and where $r_{it}$ and $w_{it}$ are the rental rate of capital and the wage rate in industry $i$ at time $t$, respectively.

\paragraph{Productivity growth accounting.} Combining the above equations, we can express aggregate TFP growth as:
\begin{align*}
    \mathrm{d}\ln(A_t) &= \sum_{i=1}^N b_{it} \mathrm{d}\ln(A_{it}) \\
    &+ \sum_{i=1}^N (b_{it} \alpha_{it}^K - \omega_{it}^K \lambda_t^K) \mathrm{d}\ln(K_{it}) \\
    &+ \sum_{i=1}^N (b_{it} \alpha_{it}^L - \omega_{it}^L \lambda_t^L) \mathrm{d}\ln(L_{it}).
\end{align*}
Following \citet{Halperin_Mazlish_2024}, adding and subtracting the term $\sum_{i=1}^N b_{it_0} \mathrm{d}\ln(A_{it})$ and summing from time $t_0$ to $t_1$, we can further decompose the cumulative aggregate TFP growth between these two dates into the following components:
\begin{alignat}{2}
    \sum_{t=t_0}^{t_1} \mathrm{d}\ln(A_t) &= \sum_{t=t_0}^{t_1} \sum_{i=1}^N b_{it_0} \mathrm{d}\ln(A_{it}) &&\quad \text{\textcolor{HEC}{Productivity}} \label{eq:productivity_1} \\
    &+ \sum_{t=t_0}^{t_1} \sum_{i=1}^N (b_{it} - b_{it_0}) \mathrm{d}\ln(A_{it}) &&\quad \text{\textcolor{HEC}{Baumol}} \label{eq:baumol_1} \\
    &+ \sum_{t=t_0}^{t_1} \sum_{i=1}^N (b_{it} \alpha_{it}^K - \omega_{it}^K \lambda_t^K) \mathrm{d}\ln(K_{it}) &&\quad \text{\textcolor{HEC}{Capital}} \label{eq:capital} \\
    &+ \sum_{t=t_0}^{t_1} \sum_{i=1}^N (b_{it} \alpha_{it}^L - \omega_{it}^L \lambda_t^L) \mathrm{d}\ln(L_{it}) &&\quad \text{\textcolor{HEC}{Labor.}} \label{eq:labor}
\end{alignat}
The interpretation of these four terms is as follows.

\paragraph{1. Within-industry productivity.} Term \eqref{eq:productivity_1} weighs industry-level TFP growth by each industry's initial share of value-added. That is, if we freeze each industry's share of nominal value-added, we capture the contribution of within-industry productivity improvements alone--how much aggregate TFP has grown purely because each industry became more productive over time.

\paragraph{2. The Baumol effect.} Term \eqref{eq:baumol_1} instead weighs industry-level TFP growth by the change in each industry's share of value-added between dates $t_0$ and $t_1$. Therefore, it measures how the changing industrial composition of the economy contributes to aggregate TFP growth. This term is negative when industries with low productivity growth become more important in the economy, or vice-versa.

\paragraph{3. Factor reallocation.} Terms \eqref{eq:capital} and \eqref{eq:labor} capture the contribution of reallocating factors across sectors. It weighs changes in industry-level capital and labor inputs by the difference between two terms. The first term is the product of the industry's share of total value-added and the elasticity of its output with respect to inputs. The second term is the product of the industry's share of total factor expenditures and the elasticity of aggregate output with respect to inputs. Intuitively, moving inputs into industries where they are more productive than the economy-wide average ($\alpha_{it}^K > \lambda_t^K$ or $\alpha_{it}^L > \lambda_t^L$) and where the receiving industries are more important in output than costs ($b_{it} > \omega_{it}^K$ or $b_{it} > \omega_{it}^L$) (i.e., high markup industries) raises aggregate productivity growth.

\begin{comment}
\citet{Baqaee_Farhi_2019} also show that aggregate TFP growth in inefficient economies can be approximated as:
\begin{equation*}
    \mathrm{d}\ln(A_t) = \underbrace{\sum_{i=1}^N \lambda_{it} \mathrm{d}\ln(A_{it})}_{\text{Technology}} - \underbrace{\sum_{i=1}^N \lambda_{it} \mathrm{d}\ln(\mu_{it}) - \lambda_t^K \mathrm{d}\ln(\Gamma_t^K) - \lambda_t^L \mathrm{d}\ln(\Gamma_t^L)}_{\text{Allocative efficiency}}
\end{equation*}
where $\mu_{it}$ is the wedge between the price of industry $i$'s output and the cost of its inputs, and $\Gamma_t^K$ and $\Gamma_t^L$ are the aggregate capital and labor shares of value-added, respectively. They decompose aggregate TFP growth into a technology component and an allocative efficiency component. However, since wedges are not directly observable, we treat them as residuals using the definition of aggregate TFP growth above.

Following the same logic as above, we can further decompose cumulative aggregate TFP growth between two dates $t_0$ and $t_1$ into the following components:
\begin{alignat}{2}
    &\sum_{t=t_0}^{t_1} \mathrm{d}\ln(A_t) = \sum_{t=t_0}^{t_1} \sum_{i=1}^N \lambda_{it_0} \mathrm{d}\ln(A_{it}) &&\quad \text{\textcolor{HEC}{Productivity}} \label{eq:productivity_2} \\
    &+ \sum_{t=t_0}^{t_1} \sum_{i=1}^N (\lambda_{it} - \lambda_{it_0}) \mathrm{d}\ln(A_{it}) &&\quad \text{\textcolor{HEC}{Baumol}} \label{eq:baumol_2} \\
    &- \sum_{t=t_0}^{t_1} \sum_{i=1}^N \lambda_{it} \mathrm{d}\ln(\mu_{it}) - \lambda_t^K \mathrm{d}\ln(\Gamma_t^K) - \lambda_t^L \mathrm{d}\ln(\Gamma_t^L) &&\quad \text{\textcolor{HEC}{Allocative efficiency.}} \label{eq:misallocation}
\end{alignat}
\end{comment}

\section{Data}
\label{s:data}

Our empirical analysis covers the Canadian economy over the period 1961 to 2019. To implement the decomposition described in Section \ref{s:theoretical framework}, we construct a dataset combining industry-level productivity measures with input-output linkages at the 3-digit North American Industry Classification System (NAICS) level. The data are drawn from two primary Statistics Canada sources.

\paragraph{Productivity and factor inputs.} We use industry-level production data from Statistics Canada Table 36-10-0217-01 (Multifactor productivity, value-added, capital input and labour input in the aggregate business sector and major sub-sectors). This table provides the core variables required for the growth accounting components of our decomposition:
\begin{itemize}
    \item \textbf{Output and TFP:} We measure nominal value-added ($P_{it} Y_{it}$) using Gross Domestic Product at basic prices. For technical change, we utilize the index of Multifactor Productivity ($A_{it}$) based on value-added.
    \item \textbf{Factor inputs:} We obtain indices for capital input ($K_{it}$) and labor input ($L_{it}$) to track real factor growth.
    \item \textbf{Factor weights:} To construct the required cost shares and elasticities, we use data on capital cost ($r_{it} K_{it}$) and labor compensation ($w_{it} L_{it}$).
\end{itemize}

\paragraph{Input-output linkages.} To calibrate the input-output architecture of the economy, we utilize the Symmetric Input-Output Tables from Statistics Canada Table 36-10-0001-01 and its historical predecessors. These tables provide the intermediate input expenditures ($\{P_{jt} X_{ijt}\}_{j=1}^N$) between industries. We use these flows to construct the cost-based input-output matrix $\bm{\Omega}_t$ and the resulting Leontief inverse $\bm{\Psi}_t$ for each year in our sample. By harmonizing these two data sources at the 3-digit NAICS level, we obtain a balanced panel of $N$ industries spanning nearly six decades, allowing us to track the evolution of aggregate TFP through the lens of the \citet{Baqaee_Farhi_2019} framework.

\begin{comment}
To calculate and decompose aggregate TFP growth in Canada, we need data on the following variables:
\begin{enumerate}
    \item Industry-level nominal value-added ($\{P_{it} Y_{it}\}_{i=1}^N$): We use data from 1961 to 2019 on gross domestic product at the 3-digit NAICS level from \href{https://www150.statcan.gc.ca/t1/tbl1/en/tv.action?pid=3610021701}{Table 36-10-0217-01} of Statistics Canada.
    \item Industry-level total-factor productivity ($\{A_{it}\}_{i=1}^N$): We use data from 1961 to 2019 on multifactor productivity based on value-added at the 3-digit NAICS level from \href{https://www150.statcan.gc.ca/t1/tbl1/en/tv.action?pid=3610021701}{Table 36-10-0217-01} of Statistics Canada.
    \item Industry-level capital ($\{K_{it}\}_{i=1}^N$): We use data from 1961 to 2019 on capital inputs at the 3-digit NAICS level from \href{https://www150.statcan.gc.ca/t1/tbl1/en/tv.action?pid=3610021701}{Table 36-10-0217-01} of Statistics Canada.
    \item Industry-level labor ($\{L_{it}\}_{i=1}^N$): We use data from 1961 to 2019 on labor inputs at the 3-digit NAICS level from \href{https://www150.statcan.gc.ca/t1/tbl1/en/tv.action?pid=3610021701}{Table 36-10-0217-01} of Statistics Canada.
    \item Industry-level capital expenditures ($\{r_{it} K_{it}\}_{i=1}^N$): We use data from 1961 to 2019 on capital cost at the 3-digit NAICS level from \href{https://www150.statcan.gc.ca/t1/tbl1/en/tv.action?pid=3610021701}{Table 36-10-0217-01} of Statistics Canada.
    \item Industry-level labor expenditures ($\{w_{it} L_{it}\}_{i=1}^N$): We use data from 1961 to 2019 on labor compensation at the 3-digit NAICS level from \href{https://www150.statcan.gc.ca/t1/tbl1/en/tv.action?pid=3610021701}{Table 36-10-0217-01} of Statistics Canada.
    \item Industry-level intermediate input expendiures ($\{\{P_{jt} X_{ijt}\}_{j=1}^N\}_{i=1}^N$): We use data from 1961 to 2019 on the symmetric input-output tables at the 3-digit NAICS level from \href{https://www150.statcan.gc.ca/t1/tbl1/en/tv.action?pid=3610000101}{Table 36-10-0001-01} of Statistics Canada and its previous versions.
\end{enumerate}
\end{comment}

\section{Empirical results}
\label{s:empirical results}

\begin{figure}[h]
    \centering
    \caption{Price and TFP Growth}
    \includegraphics[width=0.7\textwidth]{price_tfp_growth.png}
    \label{fig:price_tfp_growth}
\end{figure}

\begin{figure}[h]
    \centering
    \caption{Wage and TFP Growth}
    \includegraphics[width=0.7\textwidth]{wage_tfp_growth.png}
    \label{fig:wage_tfp_growth}
\end{figure}

\begin{figure}[h]
    \centering
    \caption{Nominal GDP and TFP Growth}
    \includegraphics[width=0.7\textwidth]{va_tfp_growth.png}
    \label{fig:va_tfp_growth}
\end{figure}

\begin{figure}[h]
    \centering
    \caption{Real GDP and TFP Growth}
    \includegraphics[width=0.7\textwidth]{real_va_tfp_growth.png}
    \label{fig:real_va_tfp_growth}
\end{figure}

\begin{figure}[h]
    \centering
    \caption{Input Cost Shares and TFP Growth}
    \begin{subfigure}[b]{0.48\textwidth}
        \centering
        \caption{Capital}
        \includegraphics[width=\textwidth]{capital_share_tfp_growth.png}
        \label{fig:capital_share_tfp_growth}
    \end{subfigure}
    \hfill
    \begin{subfigure}[b]{0.48\textwidth}
        \centering
        \caption{Labor}
        \includegraphics[width=\textwidth]{labor_share_tfp_growth.png}
        \label{fig:labor_share_tfp_growth}
    \end{subfigure}
    \label{fig:input_share_tfp_growth}
\end{figure}

\includetables{tfp_decomposition_1}

%\includetables{tfp_decomposition_2}

\begin{figure}[h]
    \centering
    \caption{TFP Growth and the Baumol Effect}
    \includegraphics[width=0.7\textwidth]{baumol.png}
    \label{fig:baumol}
\end{figure}

\begin{figure}[h]
    \centering
    \caption{TFP Growth Decomposition}
    \includegraphics[width=0.7\textwidth]{tfp_decomposition_1.png}
    \label{fig:tfp_decomposition_1}
\end{figure}

\begin{comment}
\begin{figure}[h]
    \centering
    \caption{TFP Growth and Misallocation}
    \includegraphics[width=0.7\textwidth]{misallocation.png}
    \label{fig:misallocation}
\end{figure}

\begin{figure}[h]
    \centering
    \caption{TFP Growth Decomposition \#2}
    \includegraphics[width=0.7\textwidth]{tfp_decomposition_2.png}
    \label{fig:tfp_decomposition_2}
\end{figure}
\end{comment}

\begin{figure}[h]
    \centering
    \caption{Industrial Contributions to TFP Growth}
    \includegraphics[width=0.9\textwidth]{tfp_contribution_industry.png}
    \label{fig:tfp_contribution_industry}
\end{figure}

\begin{figure}[h]
    \centering
    \caption{TFP Growth by Industry}
    \includegraphics[width=0.9\textwidth]{tfp_growth_industry.png}
    \label{fig:tfp_growth_industry}
\end{figure}

\begin{figure}[h]
    \centering
    \caption{GDP Growth by Industry}
    \includegraphics[width=0.9\textwidth]{va_growth_industry.png}
    \label{fig:va_growth_industry}
\end{figure}

\section{Conclusion}
\label{s:conclusion}

\clearpage

\bibliography{references}
\bibliographystyle{aer}

\clearpage
\renewcommand{\contentsname}{Appendix}
\addtocontents{toc}{\protect\setcounter{tocdepth}{2}}
\tableofcontents

\clearpage
\appendix
\numberwithin{equation}{section}
\renewcommand\thefigure{\thesection.\arabic{figure}}
\renewcommand\thetable{\thesection.\arabic{table}}

\section{Empirical Appendix}

\includetables{tfp_decomposition_1_no_oge}

%\includetables{tfp_decomposition_2_no_oge}

\begin{figure}[h]
    \centering
    \caption{TFP Growth and the Baumol Effect (without O\&G)}
    \includegraphics[width=0.7\textwidth]{baumol_no_oge.png}
    \label{fig:baumol_no_oge}
\end{figure}

\begin{comment}
\begin{figure}[h]
    \centering
    \caption{TFP Growth and Misallocation (without O\&G)}
    \includegraphics[width=0.7\textwidth]{misallocation_no_oge.png}
    \label{fig:misallocation_no_oge}
\end{figure}
\end{comment}

\begin{figure}[h]
    \centering
    \caption{Prices, Wages, and TFP Growth}
    \begin{subfigure}[b]{0.48\textwidth}
        \centering
        \caption{Prices}
        \includegraphics[width=\textwidth]{price_tfp_growth_period.png}
        \label{fig:price_tfp_growth_period}
    \end{subfigure}
    \hfill
    \begin{subfigure}[b]{0.48\textwidth}
        \centering
        \caption{Wages}
        \includegraphics[width=\textwidth]{wage_tfp_growth_period.png}
        \label{fig:wage_tfp_growth_period}
    \end{subfigure}
    \label{fig:price_wage_tfp_growth_period}
\end{figure}

\begin{figure}[h]
    \centering
    \caption{GDP and TFP Growth}
    \begin{subfigure}[b]{0.48\textwidth}
        \centering
        \caption{Nominal GDP}
        \includegraphics[width=\textwidth]{va_tfp_growth_period.png}
        \label{fig:va_tfp_growth_period}
    \end{subfigure}
    \hfill
    \begin{subfigure}[b]{0.48\textwidth}
        \centering
        \caption{Real GDP}
        \includegraphics[width=\textwidth]{real_va_tfp_growth_period.png}
        \label{fig:real_va_tfp_growth_period}
    \end{subfigure}
    \label{fig:gdp_tfp_growth_period}
\end{figure}

\end{document}
